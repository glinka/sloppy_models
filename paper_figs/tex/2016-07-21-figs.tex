\documentclass[11pt]{article}

\usepackage{graphicx, subcaption, amsfonts, amsmath, amsthm, empheq, setspace, lscape, xcolor}
\usepackage[top=1in, bottom=1in, left=1in, right=1in]{geometry}

% define some commands
% command to box formula
\newcommand*\widefbox[1]{\fbox{\hspace{2em}#1\hspace{2em}}}
\newlength\dlf
\newcommand\alignedbox[2]{
  % Argument #1 = before & if there were no box (lhs)
  % Argument #2 = after & if there were no box (rhs)
  &  % Alignment sign of the line
  {
    \settowidth\dlf{$\displaystyle #1$}  
    % The width of \dlf is the width of the lhs, with a displaystyle font
    \addtolength\dlf{\fboxsep+\fboxrule}  
    % Add to it the distance to the box, and the width of the line of the box
    \hspace{-\dlf}  
    % Move everything dlf units to the left, so that & #1 #2 is aligned under #1 & #2
    \boxed{#1 #2}
    % Put a box around lhs and rhs
  }
}

\newcommand\ER{Erd\H{o}s-R'{e}nyi}
\newcommand{\Forall}{\; \forall \;}
\DeclareMathOperator*{\argmin}{\arg\!\min}

% change captions
\captionsetup{width=0.8\textwidth}
\captionsetup{labelformat=empty,labelsep=none}

% set paragraph indent length
\setlength\parindent{0pt}

% set folder for imported graphics
\graphicspath{ {../figs/} }

\title{More paper figs}
\author{}

\begin{document}
\maketitle

\begin{figure}[htbp]
  \centering
  \includegraphics[width=\linewidth]{./linear-sing-pert/traditional-mm}
  \caption{Parameter space (left) and corresponding model manifold
    (right) for the basic singularly perturbed example $x' =
    x/\epsilon$. Here we indicate $x_0$ with color and $\epsilon$ with
  vertical lines. Both parameter space and the model manifold are cut
  so that the colored areas represent a (nearly) one-to-one mapping
  between each other. Previous plots included regions in parameter
  space that were not represented on the model manifold, and likewise
  sections of the model manifold that weren't shown in parameter
  space. Here the mapping is nearly one-to-one between left
  and right figures.}
\end{figure}

\begin{figure}[htbp]
  \centering
  \includegraphics[width=\linewidth]{./linear-sing-pert/twod-cmap-mm}
  \caption{This is a very similar figure to that above, but now we use a sort of
  two-dimensional colormap that indicates simultaneously the $x_0$ and
  $\epsilon$ values.}
\end{figure}

\begin{figure}[htbp]
  \centering
  \includegraphics[width=\linewidth]{./model-manifold/mm-ps-illustration}
  \caption{Here we have an updated illustration of the parameter space
    contours opening up to infinity when the model manifold boundary
    is crossed. The gradient from white to color is meant to show the
    correspondence between the two plots. I'm not sure this would be
    necessary in the paper as we have a very simple, data-based
    example in the singularly perturbed system mentioned above.}
\end{figure}


\begin{landscape}
  
  \begin{figure}[htbp]
    \centering
    \includegraphics[width=\linewidth]{./rawlings/keff-delta-complete-v2}
    \caption{An illustration of how the region in parameter space that
    falls within increasing values of $\delta(\theta)$ and increasing
    ranges of $k_{eff}$ match one another. From left to right
    $\delta(\theta)$ increases, while the range of $k_{eff}$ also
    grows. This is illustrated by the line plots underneath each 3d
    figure. The final figure on the right overlays both
    $\delta(\theta)$ and $k_{eff}$ revealing how closely they match,
    thus showing that $\delta(\theta)$ changes with $k_{eff}$.}
  \end{figure}
  

\end{landscape}

\end{document}
