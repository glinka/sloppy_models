\documentclass[11pt]{article}

\usepackage{graphicx, subcaption, amsfonts, amsmath, amsthm, empheq, setspace, lscape, xcolor}
\usepackage[top=1in, bottom=1in, left=1in, right=1in]{geometry}

% define some commands
% command to box formula
\newcommand*\widefbox[1]{\fbox{\hspace{2em}#1\hspace{2em}}}
\newlength\dlf
\newcommand\alignedbox[2]{
  % Argument #1 = before & if there were no box (lhs)
  % Argument #2 = after & if there were no box (rhs)
  &  % Alignment sign of the line
  {
    \settowidth\dlf{$\displaystyle #1$}  
    % The width of \dlf is the width of the lhs, with a displaystyle font
    \addtolength\dlf{\fboxsep+\fboxrule}  
    % Add to it the distance to the box, and the width of the line of the box
    \hspace{-\dlf}  
    % Move everything dlf units to the left, so that & #1 #2 is aligned under #1 & #2
    \boxed{#1 #2}
    % Put a box around lhs and rhs
  }
}

\newcommand\ER{Erd\H{o}s-R'{e}nyi}
\newcommand{\Forall}{\; \forall \;}
\DeclareMathOperator*{\argmin}{\arg\!\min}

% change captions
\captionsetup{width=0.8\textwidth}
\captionsetup{labelformat=empty,labelsep=none}

% set paragraph indent length
\setlength\parindent{0pt}

% set folder for imported graphics
\graphicspath{ {../figs/} }

\title{Nonlinear transformation of parameter space}

\begin{document}
\maketitle

I used Antonios' model, investigating behavior in a ball in dataspace
around the point corresponding to

\begin{align*}
  a &= 1 \\
  b &= 0.01 \\
  \lambda &= 1 \\
  \epsilon &= 0.001
\end{align*}

Taking constant values for the initial conditions, $b$, and $\epsilon$
and investigating the values of $a$ and $\lambda$ that keep the model
response within a certain distance of the true data (here we looked at
the ball $\| f(a, \lambda) - d \| < 50$), we find that the following
region of parameter space maps to points within this ball. Please
forgive the clipping of the ellipse, it'll be remedied in future
plots. \\

\begin{figure}[htbp]
  \centering
  \includegraphics[width=\linewidth]{original-params}
  \caption{Preimage of intersection of ball with manifold, an unsloppy
  ellipse}
\end{figure}

Successive applications of the H\'{e}non map produce an increasingly
sloppy set of points as shown in subsequent figures. Color indicates
distance from original data in dataspace.


\begin{figure}[htbp]
  \centering
  \includegraphics[width=\linewidth]{one-iter-params}
  \caption{Transformation of parameter set by one iteration of H\'{e}non map}
\end{figure}


\begin{figure}[htbp]
  \centering
  \includegraphics[width=\linewidth]{two-iter-params}
  \caption{Transformation of parameter set by two iterations of H\'{e}non map}
\end{figure}

\begin{figure}[htbp]
  \centering
  \includegraphics[width=\linewidth]{two-iter-params}
  \caption{Transformation of parameter set by two iterations of H\'{e}non map}
\end{figure}

\begin{figure}[htbp]
  \centering
  \includegraphics[width=\linewidth]{three-iter-params}
  \caption{Transformation of parameter set by three iterations of H\'{e}non map}
\end{figure}

\begin{figure}[htbp]
  \centering
  \includegraphics[width=\linewidth]{four-iter-params}
  \caption{Transformation of parameter set by four iterations of H\'{e}non map}
\end{figure}

% \bibliographystyle{plain}
% \bibliography{literature.bib}

\end{document}
