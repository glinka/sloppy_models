\documentclass[11pt]{article}

\usepackage{graphicx, subcaption, amsfonts, amsmath, amsthm, empheq, setspace, lscape, xcolor}
\usepackage[top=1in, bottom=1in, left=1in, right=1in]{geometry}

% define some commands
% command to box formula
\newcommand*\widefbox[1]{\fbox{\hspace{2em}#1\hspace{2em}}}
\newlength\dlf
\newcommand\alignedbox[2]{
  % Argument #1 = before & if there were no box (lhs)
  % Argument #2 = after & if there were no box (rhs)
  &  % Alignment sign of the line
  {
    \settowidth\dlf{$\displaystyle #1$}  
    % The width of \dlf is the width of the lhs, with a displaystyle font
    \addtolength\dlf{\fboxsep+\fboxrule}  
    % Add to it the distance to the box, and the width of the line of the box
    \hspace{-\dlf}  
    % Move everything dlf units to the left, so that & #1 #2 is aligned under #1 & #2
    \boxed{#1 #2}
    % Put a box around lhs and rhs
  }
}

\newcommand\ER{Erd\H{o}s-R'{e}nyi}
\newcommand{\Forall}{\; \forall \;}
\DeclareMathOperator*{\argmin}{\arg\!\min}

% change captions
\captionsetup{width=0.8\textwidth}
\captionsetup{labelformat=empty,labelsep=none}

% set paragraph indent length
\setlength\parindent{0pt}

% set folder for imported graphics
\graphicspath{ {../figs/} }

\title{H\'{e}non map and anisotropic DMAPS figures}

\begin{document}
\maketitle

\section{H\'{e}non map figures}

\begin{figure}[htbp]
  \centering
  \includegraphics[width=\linewidth]{./transformed-params/henon-combined}
  \caption{Here we show a dataset found through optimization (left)
    and its image under two inverse H\'{e}non mappings (right),
    illustrating that properly chosen nonlinear transformations may
  remove sloppiness. We use Antonios' model here.}
\end{figure}

\section{Anisotropic DMAPS figures}


\begin{figure}[htbp]
  \centering
  \includegraphics[width=\linewidth]{./rawlings/data-dmaps-keff-a}
  \caption{Here we use the reversible reaction system, and look at a
    collection of parameter combinations that predict concentration
    profiles close to some ``base'' profile. We allow large enough
    deviations to see some fattening of the dataset, and apply the new
  DMAPS with a large value of $\epsilon$ and small $\lambda$. The
  first eigenvector of this embedding aligns well with $k_{eff}$ as
  hoped. The dataset is colored by $\Phi_1$ and two surfaces of
  constant $k_{eff}$ are added to show how the level sets of $\Phi_1$
  and the level sets of $k_{eff}$ coincide.}
\end{figure}

\begin{figure}[htbp]
  \centering
  \includegraphics[width=\linewidth]{./rawlings/phi1-keff}
  \caption{Here we directly plot $\Phi_1$ against $k_{eff}$ to show
    that they are approximately one-to-one.}
\end{figure}

% \bibliographystyle{plain}
% \bibliography{literature.bib}

\end{document}
