\documentclass[11pt]{article}

\usepackage{graphicx, subcaption, amsfonts, amsmath, amsthm, empheq,
  setspace, lscape}
\usepackage[top=1in, bottom=1in, left=1in, right=1in]{geometry}

% define some commands
% command to box formula
\newcommand*\widefbox[1]{\fbox{\hspace{2em}#1\hspace{2em}}}
\newlength\dlf
\newcommand\alignedbox[2]{
  % Argument #1 = before & if there were no box (lhs)
  % Argument #2 = after & if there were no box (rhs)
  &  % Alignment sign of the line
  {
    \settowidth\dlf{$\displaystyle #1$}  
    % The width of \dlf is the width of the lhs, with a displaystyle font
    \addtolength\dlf{\fboxsep+\fboxrule}  
    % Add to it the distance to the box, and the width of the line of the box
    \hspace{-\dlf}  
    % Move everything dlf units to the left, so that & #1 #2 is aligned under #1 & #2
    \boxed{#1 #2}
    % Put a box around lhs and rhs
  }
}

\newcommand{\ps}{\mathrm{\Theta}}
\newcommand{\p}{\theta}
\newcommand{\eps}{\varepsilon}
\newcommand{\be}{\begin{equation}}
\newcommand{\ee}{\end{equation}}
\newcommand\ER{Erd\H{o}s-R'{e}nyi}
\newcommand{\Forall}{\; \forall \;}
\DeclareMathOperator*{\argmin}{\arg\!\min}

% change captions
\captionsetup{width=0.8\textwidth}
% \captionsetup{labelformat=empty,labelsep=none} 

% set paragraph indent length
\setlength\parindent{0pt}

% set folder for imported graphics
\graphicspath{ {../figs/} }

\title{Opening of contours in parameter space}

\begin{document}
\maketitle

The system is the incredibly basic, singularly perturbed ODE

\begin{align}
  y' = -\frac{1}{\epsilon} y
  \label{eq:sp}
\end{align}

We allow $y_0$ and $\epsilon$ to vary and measure

\begin{align}
  \mu(\theta) = \begin{bmatrix} y_0 \exp(-\frac{t_1}{\epsilon}) \\ \\ y_0
    \exp(-\frac{t_2}{\epsilon}) \\ \\ y_0
    \exp(-\frac{t_3}{\epsilon}) \end{bmatrix}
  \label{eq:mr}
\end{align}

As $\frac{1}{\epsilon} \rightarrow 0$ we see that $y(t) = y_0 \;
\forall t$ corresponding to a model manifold boundary at $y_1 = y_2 =
y_3$. For $\frac{1}{\epsilon} \rightarrow \infty$ we have
$y(t) \rightarrow 0 \; \forall \; t$ which corresponds to a boundary
of $y_2 = y_3 = 0$. We can understand the second boundary by
considering that for any $\frac{1}{\epsilon} > 0$, we can find a $y_0$
such that $y(t_1)$ achieves any desired value. This still holds as
$\frac{1}{\epsilon} \rightarrow \infty$, but in this regime $y_2
\rightarrow y_3 \rightarrow 0$. These boundaries are clearly visible
in Fig (\ref{fig:lmm}).

\begin{figure}[htbp]
  \centering
  \includegraphics[width=\linewidth]{./model-manifold/linear-sing-pert-mm}
  \caption{Model manifold $\mathcal{M}$ for dynamical
    system~\eqref{eq:sp} and model response~\eqref{eq:mr}.
(Left) Parameter plane with lines of constant $\eps$ and coloring by $y_0$.
(Right) $3$D data space with embedded $2$D model surface.
Each pair $(\eps,y_0)$ is mapped to a point on $\mathcal{M}$ through $\mathbf{f}$.
The coloring is also by $y_0$ and lines emanating from the origin have constant $\eps$ along them.
The surface is bounded by the diagonal $f_1=f_2=f_3$ ($\eps\to\infty$)
and the $f_1-$axis ($\eps\downarrow0$) but unbounded along lines
$\eps=constant$. The mirror image of the surface through the origin (not shown) also
belongs to $\mathcal{M}$ ($y_0<0$), as does the grey part above the
diagonal ($\eps<0$). \label{fig:lmm}}
\end{figure}

\begin{figure}
  \centering
  \includegraphics[width=\linewidth]{./model-manifold/lsp-paramspace-ii}
  \caption{Examining level curves in parameter space. We choose a
    point on the model manifold close to the $f_1$ axis and look at
    what parameters are some distance $\delta$ from that point (using
    Euclidean distance in parameter space and not distance along the
    manifold). At first we see ellipsoidal curves, as expected for
    small $\delta$, but once we cross the model manifold boundary
    at larger $\delta$ the curves open up. \label{fig:lmm-lc} }
\end{figure}


\end{document}
