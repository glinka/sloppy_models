\documentclass[11pt]{article}

\usepackage{graphicx, subcaption, amsfonts, amsmath, amsthm, empheq, setspace, lscape, xcolor}
\usepackage[top=1in, bottom=1in, left=1in, right=1in]{geometry}

% define some commands
% command to box formula
\newcommand*\widefbox[1]{\fbox{\hspace{2em}#1\hspace{2em}}}
\newlength\dlf
\newcommand\alignedbox[2]{
  % Argument #1 = before & if there were no box (lhs)
  % Argument #2 = after & if there were no box (rhs)
  &  % Alignment sign of the line
  {
    \settowidth\dlf{$\displaystyle #1$}  
    % The width of \dlf is the width of the lhs, with a displaystyle font
    \addtolength\dlf{\fboxsep+\fboxrule}  
    % Add to it the distance to the box, and the width of the line of the box
    \hspace{-\dlf}  
    % Move everything dlf units to the left, so that & #1 #2 is aligned under #1 & #2
    \boxed{#1 #2}
    % Put a box around lhs and rhs
  }
}

\newcommand\ER{Erd\H{o}s-R'{e}nyi}
\newcommand{\Forall}{\; \forall \;}
\DeclareMathOperator*{\argmin}{\arg\!\min}

% change captions
\captionsetup{width=0.8\textwidth}
\captionsetup{labelformat=empty,labelsep=none}

% set paragraph indent length
\setlength\parindent{0pt}

% set folder for imported graphics
\graphicspath{ {../figs/} }

\title{Data-space kernel results}

\begin{document}
\maketitle

\section{Background}

We investigate the DMAPS kernel

\begin{align*}
  k(\theta_i, \theta_j) = e^{\frac{\|\theta_i - \theta_j\|^2}{\epsilon} - \alpha \frac{\|\mu(\theta_i) - \mu(\theta_j)\|^2}{\epsilon^2}}
\end{align*}

where $\theta$ is some parameter combination and $\mu(\theta)$ is the
model prediction at $\theta$. In order to better understand the
anisotropic diffusion implied by this function, we vary $\alpha$ and
examine the resulting embeddings. $\epsilon$ was fixed to $10^{-1}$
throughout; future tests will examine different values.

The system in question consists of both the domain and range of $f:
\mathbb{R}^2 \rightarrow \mathbb{R}^2$. Specifically, we take $f$ to
be a linear function given by

\begin{align*}
  f = V \Lambda V^{-1}
\end{align*}

with

\begin{align*}
  \Lambda = \begin{bmatrix} \lambda_1 & 0 \\ 0 & \lambda_2 \end{bmatrix}
\end{align*}
  
and

\begin{align*}
  V = \begin{bmatrix} \cos(\theta_1) & \cos(\theta_2) \\ \sin(\theta_1)
    & \sin(\theta_2) \end{bmatrix}
\end{align*}

We take $\lambda_1 = 1$, $\lambda_2 = 10^{-1}$, $\theta_1 =
\frac{\pi}{4}$ and $\theta_2 = \frac{3\pi}{4}$. When applied to the
unit square, shown in blue in the figure below, the resulting image is
the red parallelogram. This four-dimensional dataset consisting of
$(x_i, y_i, f_1(x_i, y_i), f_2(x_i, y_i))$ was used throughout the
experiments below.

\begin{figure}[htbp]
  \centering
  \includegraphics[width=\linewidth]{linear-dataset}
  \caption{The preimage (blue square) and image (red parallelogram),
    which combined consist of the DMAPS dataset}
\end{figure}

\clearpage

\section{Results}

\subsection{$\alpha=0$}

First we check $\alpha = 0$ for sanity's sake. We indeed recover the
parameterization of the preimage, the simple square.

\begin{figure}[htbp]
  \centering
  \includegraphics[width=\linewidth]{./dmaps-dataspace-alpha01.png}
  \caption{Coloring of dataset by first eigenvector with $\alpha=0$}
\end{figure}

\begin{figure}[htbp]
  \centering
  \includegraphics[width=\linewidth]{./dmaps-dataspace-alpha02.png}
  \caption{Coloring of dataset by second eigenvector with $\alpha=0$}
\end{figure}

\clearpage

\subsection{$\alpha=10^{-8}$}

Increasing to $\alpha = 10^{-8}$ we find, first, a parameterization of
the image, and later in the spectrum a parameterization of the
preimage. 

\begin{figure}[htbp]
  \centering
  \includegraphics[width=\linewidth]{./dmaps-dataspace-alpha81.png}
  \caption{Coloring of dataset by first eigenvector with $\alpha=0$}
\end{figure}

\begin{figure}[htbp]
  \centering
  \includegraphics[width=\linewidth]{./dmaps-dataspace-alpha82.png}
  \caption{Coloring of dataset by second eigenvector with $\alpha=0$}
\end{figure}

\begin{figure}[htbp]
  \centering
  \includegraphics[width=\linewidth]{./dmaps-dataspace-alpha815.png}
  \caption{Coloring of dataset by fifteenth eigenvector with $\alpha=0$}
\end{figure}

\begin{figure}[htbp]
  \centering
  \includegraphics[width=\linewidth]{./dmaps-dataspace-alpha816.png}
  \caption{Coloring of dataset by sixteenth eigenvector with $\alpha=0$}
\end{figure}

\clearpage

\subsection{$\alpha=1$}

Increasing $\alpha$ any further yields only a parameterization of the
image within the first eighty calculated eigenvectors. The results
below follow from $\alpha=1$.

\begin{figure}[htbp]
  \centering
  \includegraphics[width=\linewidth]{./dmaps-dataspace-alpha81.png}
  \caption{Coloring of dataset by first eigenvector with $\alpha=0$}
\end{figure}

\begin{figure}[htbp]
  \centering
  \includegraphics[width=\linewidth]{./dmaps-dataspace-alpha82.png}
  \caption{Coloring of dataset by second eigenvector with $\alpha=0$}
\end{figure}

% \bibliographystyle{plain}
% \bibliography{literature.bib}

\end{document}
