\documentclass[11pt]{article}

\usepackage{graphicx, subcaption, amsfonts, amsmath, amsthm, empheq, setspace, lscape, xcolor}
\usepackage[top=1in, bottom=1in, left=1in, right=1in]{geometry}

% define some commands
% command to box formula
\newcommand*\widefbox[1]{\fbox{\hspace{2em}#1\hspace{2em}}}
\newlength\dlf
\newcommand\alignedbox[2]{
  % Argument #1 = before & if there were no box (lhs)
  % Argument #2 = after & if there were no box (rhs)
  &  % Alignment sign of the line
  {
    \settowidth\dlf{$\displaystyle #1$}  
    % The width of \dlf is the width of the lhs, with a displaystyle font
    \addtolength\dlf{\fboxsep+\fboxrule}  
    % Add to it the distance to the box, and the width of the line of the box
    \hspace{-\dlf}  
    % Move everything dlf units to the left, so that & #1 #2 is aligned under #1 & #2
    \boxed{#1 #2}
    % Put a box around lhs and rhs
  }
}

\newcommand\ER{Erd\H{o}s-R'{e}nyi}
\newcommand{\Forall}{\; \forall \;}
\DeclareMathOperator*{\argmin}{\arg\!\min}

% change captions
\captionsetup{width=0.8\textwidth}
% \captionsetup{labelformat=empty,labelsep=none}

% set paragraph indent length
\setlength\parindent{0pt}

% set folder for imported graphics
\graphicspath{ {../figs/outliers/} }

\title{Intermediate mixed kernel results and analysis of
  $k_{eff}$ outliers}

\begin{document}
\maketitle

\section{Model measurements}

Given the reaction

\begin{align*}
  A \xrightleftharpoons[k_{-1}]{k_1} B, \; B \xrightarrow[]{k_2} C
\end{align*}

for a given set of $\theta = \{k_1, k_{-1}, k_2\}$ we measure the concentration
of $C$ at five times, giving

\begin{align*}
  \mu(\theta) = \begin{bmatrix} C(0) \\ \\ C(\frac{1}{k_{eff}}) \\ \\
    C(\frac{2}{k_{eff}}) \\ \\ C(\frac{3}{k_{eff}}) \\ \\
    C(\frac{4}{k_{eff}}) \end{bmatrix}
\end{align*}

We fix $A(0) = 1$ and $B(0) = C(0) = 0$, so we could equivalently define

\begin{align*}
  \mu(\theta) = \begin{bmatrix} C(\frac{1}{k_{eff}}) \\ \\
    C(\frac{2}{k_{eff}}) \\ \\ C(\frac{3}{k_{eff}}) \\ \\
    C(\frac{4}{k_{eff}}) \end{bmatrix}
\end{align*}

as $C(0)$ is constant. \\

To generate a dataset in parameter space, we set true parameter values
of 

\begin{align*}
  k_1^* &= 1 \\
  k_{-1}^* = k_2^* &= 1000 \\
\end{align*}

and measure the \textbf{squared} distance $\| \mu(theta) -
\mu(\theta^*) \|^2$ corresponding to a least-squares objective
function. We keep points for which this distance falls beneath some
tolerance (here $10^{-3}$).

\section{Intermediate mixed kernel}

We investigate the DMAPS kernel

\begin{align}
  k(\theta_i, \theta_j) = \exp \bigg( -\frac{1}{\lambda^2} \bigg( \frac{\|\log(\theta_i)
  - \log(\theta_j)\|^2}{\epsilon^2} + \|\mu(\theta_i) -
  \mu(\theta_j)\|^2\bigg) \bigg)
  \label{eqn1}
\end{align}

with parameter values $\epsilon = 10$ and $\lambda = 1$. This smaller
value of $\lambda$ biases the diffusion along directions in which
$\mu(\theta)$ changes, i.e. along important parameter
directions. However, $\epsilon$ is not so large as to completely
remove the effects of $\| \theta_i - \theta_j \|$ on the DMAPS
results, and as a result, we find that the first two eigenvectors
$\Phi_1$ and $\Phi_2$ parameterize the sloppy directions while
$\Phi_3$ parameterizes the important direction in which $k_{eff}$
changes. This is shown in Figs. (\ref{fig1}-\ref{fig3}) below. \\

As a reminder, we have sampled the three-dimensional parameter space
$\{k_1, k_{-1}, k_2 \}$ around some true values $k_1 = 1$,
$k_{-1} = 1000$ and $k_2 = 1000$, keeping those parameter combinations that
predict a concentration profile of species $C$ within some error
tolerance (here we require $\| \mu(\theta) - \mu(\theta^*) \| <
10^{-3}$). This leaves us with the slightly thick parameter set shown
in Fig. (\ref{fig0}), which we then apply Eqn. (\ref{eqn1}) to.

\begin{figure}[htbp]
  \centering
  \includegraphics[width=\linewidth]{../rawlings/keff-large-delta}
  \caption{The dataset we will be applying DMAPS to, colored by
    $k_{eff}$. \label{fig0}}
\end{figure}

\begin{figure}[htbp]
  \centering
  \includegraphics[width=\linewidth]{./phi1}
  \caption{Dataset colored by $\Phi_1$, parameterizing a sloppy
    direction. \label{fig1}}
\end{figure}

\begin{figure}[htbp]
  \centering
  \includegraphics[width=\linewidth]{./phi2}
  \caption{Dataset colored by $\Phi_2$, parameterizing another sloppy
    direction. \label{fig2}}
\end{figure}

\begin{figure}[htbp]
  \centering
  \includegraphics[width=\linewidth]{./phi3}
  \caption{Dataset colored by $\Phi_3$, parameterizing the important
    direction, $k_{eff}$. \label{fig3}}
\end{figure}

\clearpage

\section{$k_{eff}$ outliers}

We switch now to the problem of seemingly spurious values of $k_{eff}$
that appear within the dataset. Namely, if we simply color the dataset
by $k_{eff}$ as we have done in Fig. (\ref{figk}), we see that a small
corner of the volume is colored yellow. Professor Gear suggested this
might be a result of including small contributions of
$\| \theta_i - \theta_j \|$ in the DMAPS kernel, but I believe this
behavior is inherent in the model itself as we will see. First, in
Fig. (\ref{fignc}) we plot $k_{eff}$ vs.
$\| \mu(\theta) - \mu(\theta^*) \|$, essentially looking at how well
$k_{eff}$ predicts deviations from the true trajectory. If we actually
had that $C(t) = f(k_{eff})$ as we claim with the QSSA, this figure
should yield a one-dimensional curve; however, we are apparently
sampling in regions of $\{k_1, k_{-1}, k_2\}$ in which this
approximation breaks down, and $C(t) = f(k_1, k_{-1}, k_2)$. Thus, at
every value of $k_{eff}$ we find a scattered range of values for $\| \mu(\theta) - \mu(\theta^*) \|$. \\

In Fig. (\ref{figc}) we plot the same points, but now colored by
$\Phi_1$ from the previous DMAPS output. We see that the points that
lie off the curve fall in the upper ranges of $\Phi_1$, and so in
Fig. (\ref{figrb}) we partition the points into those for which
$\Phi_1 < 0.01$ and those with $\Phi_1 \ge 0.01$, providing a nice
separation. We then examine what this corresponds to in full parameter
space, shown in Fig. (\ref{fig3dc}). The red points neatly fall into a
specific region of parameter space, generally with smaller values of
$k_2$ where one might not expect the QSSA to apply. Thus greater care
must be taken when sampling parameter space if we desire to restrict
ourselves to examining $k_{eff}$ as the single important parameter.


\begin{figure}[htbp]
  \centering
  \includegraphics[width=\linewidth]{./3d-outliers-c}
  \caption{Dataset colored by $k_{eff}$ highlighting a small region of
    abnormally high values of $k_{eff}$. \label{figk}}
\end{figure}


\begin{figure}[htbp]
  \centering
  \includegraphics[width=\linewidth]{./cost-keff-plain}
  \caption{Distance from true trajectory vs. $k_{eff}$. \label{fignc}}
\end{figure}


\begin{figure}[htbp]
  \centering
  \includegraphics[width=\linewidth]{./cost-keff-phi1}
  \caption{Same as Fig. (\ref{fignc}) but colored by $\Phi_1$. \label{figc}}
\end{figure}


\begin{figure}[htbp]
  \centering
  \includegraphics[width=\linewidth]{./cost-keff-partition}
  \caption{Partitioned based on $\Phi_1$. \label{figrb}}
\end{figure}


\begin{figure}[htbp]
  \centering
  \includegraphics[width=\linewidth]{./3d-partition}
  \caption{Dataset in partitioned by $\Phi_1$. \label{fig3dc}}
\end{figure}

\end{document}
