\documentclass{article}
\title{Data-driven identification of effective parameters
driving the evolution of multiscale systems}

\usepackage{
amsmath, amssymb, amsfonts, chemarr,
xcolor, epsf, float, subcaption
}

\usepackage[demo]{graphicx}

\setlength{\parindent}{0pt}

\newcommand{\N}{\mathrm{N}}
\newcommand{\T}{\mathrm{T}}
\newcommand{\X}{\mathrm{X}}
\newcommand{\D}{\mathrm{D}}
\newcommand{\V}{\mathcal{V}}
\newcommand{\M}{\mathcal{M}}
\newcommand{\ip}[2]{\langle#1,#2\rangle}
\newcommand{\abs}[1]{\vert#1\vert}
\newcommand{\norm}[1]{\vert\vert#1\vert\vert}
\newcommand{\dv}[2]{d(#1,#2)}
\newcommand{\eps}{\varepsilon}
\newcommand{\ssep}{:}

\newcommand{\p}{\theta}
\newcommand{\fmr}{\chi}
\newcommand{\fmm}{\mathcal{X}}
\newcommand{\fiber}{\mathcal{F}}
\newcommand{\omr}{\mu}
\newcommand{\omm}{\mathcal{M}}
\newcommand{\R}{\mathrm{R}}
\newcommand{\ps}{\mathrm{\Theta}}
\newcommand{\fms}{\mathrm{X}}

\newcommand{\be}{\begin{equation}}
\newcommand{\ee}{\end{equation}}
\newcommand{\dst}{\displaystyle}

\newcommand{\Ord}{\mathcal{O}}

\author{}

\begin{document}
\maketitle

\section{Introduction}

In a 2006 computational neuroscience paper, after repeatedly fitting a ``reasonable" mechanistic neuron model to experimental firing data, the authors arrived at a multitude of ``good enough" parameter values - values that gave acceptable agreement between model prediction and experiments in the comparison norm chosen. They then had the insightful idea to try and ``summarize" this set of ``good enough" parameter values through principal component analysis - in effect, they passed hyperplanes through their sampling of the "good enough" subset of their parameter space (good enough
for their experimental data and with their goodness criterion).
%
In effect, they created -through data mining- a parametrization of what we might call the ``zero level set" of their objective/fitness function (again, for the experimental data available and the comparison norm chosen).

Clearly, the fact that the dimensionality of this ``good set" (for short) was more than zero implied that there are
directions in parameter space (in their work, local linear combinations of the model parameters) that keep the
predictions satisfactory.
%
This then implies that the number of directions in model parameter space that actually make a difference in the predictions is less than the full parameter space dimension.
%
The model has therefore less ``effective parameters" than \em{model parameters}.

There is hardly any novelty in this observation - there have been many efforts (and in recent years notably the program on "model/parameter sloppiness" from J. Sethna's group at Cornell, continued and extended by his former student
N. Transtrum and coworkers, or the ``active subspaces" program of P. Constantine).
%
What was interesting for us in this paper, however, was the attempt to provide a quantitative description of an important aspect of sloppiness (the parametrization of the zero level set) \em{through data mining}.

In this paper, motivated by the above (and not only) we will attempt to start building a
systematic link between \em{nonlinear} data mining and ``sloppiness".
%
Our research background is in model reduction of complex (many-degree-of-freedom) dynamical systems;
our experience is therefore much more on state-variable reduction
(simplifying modeling by deriving reduced dimensionality models) - rather than in \em{parameter reduction}.
%
The context might be best illustrated by thinking of reducing a time-dependent
flow in a complex geometry - one goes from a ``huge" discretization of the Navier-Stokes equations to a
low-dimensional Galerkin projection in terms of empirical basis functions;
yet both the original problem and the "state-reduced" problem have \em{one and the same} important parameter:
the Reynolds number.
%
In contrast,when reducing a large chemical (e.g. combustion) or biochemical (e.g.metabolic) reaction network, it is clear
that one desires a model that is not only reduced in terms of effective number of variables,
but also reduced in terms of effective parameters - parameters that would arise in a
reduced dynamic model that, one expects, are (possibly linear, but in general nonlinear) functions of
combinations the (full, detailed) model parameters.
%
For several years we have been developing a framework for "on-the-fly", data based reduction approach to
modeling complex systems (what we call the equation-free approach);
we attempt to solve reduced models that are unavailable in closed form by
judicious short bursts of computation with an (assumed accurate) detailed model.
%
There is a natural link between this approach and data mining: these brief bursts
of simulation of the high dimensional detailed model produce rich, high dimensional
data-sets; mining these data (through manifold learning techniques, for our group often through diffusion maps)
suggests the reduced set of \em{effective state variables} in terms of which the reduced model should be locally cast.
%
This we call the \em{variable free} component of our approach: the reduced model variables are
determined not through experience or through mathematics, but rather dictated through data mining of model simulations.
%
Clearly, it would be naive to embark on such a process if one is capable of deriving closed form equations in
terms of explicit reduced variables determined through mathematics (e.g. a few leading moments of particle distributions,
a few "important" concentrations, a few empirical --PCA-based-- orthogonal basis function coefficients).
%
Yet, -also clearly- if one cannot succeed in such a derivation, this does not mean that one must
remain condemned to lengthy fine scale simulations and verbal/statistical descriptions of their results.
%
Knowing that an effective (reduced) equation in terms of some effective (reduced) observables/variables is
possible, can still lead to the efficient extraction of information from the fine scale model.

Our ambition is to extend this approach beyond the \em{data-driven} determination of "the right effective
variables", to the data-driven determination of "the right effective parameters".
%
This way, not only dynamical simulations can be accelerated, but also
parametric explorations and optimization computations ("dynamics" in parameter space!) can be
enhanced.


A few words before starting:

We already stated that we want to develop data-driven ways to accomplish the detection of 
\em{effective} (important, relevant) parameters or parameter combinations within a large "model parameter" set.
%
Sensitivity analysis (in particular, computer-assisted sensitivity analysis) is one of the ways this is
currently approached in many disciplines; yet it is interesting to consider the way we do this analytically 
(when possible). 
%
The appropriate context is, of course, perturbation theory: typically, one would discard a term in an expansion
if the term is relatively small; the coefficient of this term is then a sloppy parameter (any value that
keeps the term relatively small does not alter the predictions of the model). 
%
Such a coefficient may well be
a function of several model parameters (ratios of sums of kinetic constants, for example). 
%
In the context of \em{time-dependent models}, we know that there are two ``types" of perturbations:
regular, and singular. 
%
The coefficients of \em{regular} perturbation terms can be ignored (they are then ``sloppy"), and the
parameter set can thus be reduced, while the number of effective variables remains the same.
%
In the case of \em{singular} perturbations,  as illustrated by the ``standard" two-variable singularly
perturbed ODE system
......


where a sufficiently small coefficient multiplies the highest
time derivative, the sloppiness of this coefficient (and the concomitant reduction of model parameters) 
is inextricably linked to the reduction of the dimensionality of the long-term system response.
%
How do we then test numerically for ``regular" vs. ``singular" sloppiness (the latter
also linked to potential long-time model reduction) ? 
%
And while the long-term dynamics may reduce, if one has accurate measurements of initial transients, 
this could ``lift" sloppiness in both state-space and parameter-space reduction; the point here being
that it is the time scales of interest to the observer (and, therefore, the type of measurements the
observer chooses or is able to perform) that may very well affect potential model and parameter reduction.

We believe it is worth considering (in the context of the above illustrative caricature) the inclusion of the
initial condition $y_0 \equiv y(t=0)$ in the ``fast" variable $y$ as a parameter, affecting the system response, that is
to be inferred from later measurements.
%
Clearly, if one only wants to fit long-time measurements (long compared to the fast time scale $eps$) there exists a multitude of ``good enough" initial conditions (a one-parameter family, lying on the relevant iscochrone) that fit the bill.
%
This infinity (and therefore also ``sloppiness") of ``equivalent" initial conditions in \em{state space}), which is the
essential manifestation of state-space model reducibility, naturally then becomes part of \em{parameter reduction};
there is an apparent duality between sloppiness in the singular perturbation parameter $\eps$ and in our
ability to derive a long-term accurate reduced dynamic model ignoring the fast foliation of state space in the 
$y$ direction: singular parameter perturbation and model reduction are inextricably linked.
%

One can also draw a different \em{conceptual} analogy between ``dynamic state reduction" and ``parameter reduction" in the
context of optimization. 
%
Indeed, an optimization algorithm (from conjugate gradients to simulated annealing)
endows parameter space with ``search dynamics" - each iteration of the algorithm is ``time 1" in this
optimization clock. 
%
In the same sense that fast equilibrating variables can be ignored in long-term dynamics (remember again
the illustrative caricature we discussed above), sensitive parameter directions quickly ``fathomed" can
be also factored out in the ``slow", long-term optimization steps in the less-sensitive (and thus
``optimization slow") directions.
%
%Remarkably, in optimization it is the ``slow parameters", that matter less, that we are interested in 
%exploring. 
%
%In dynamics, it is the parameters who ``matter more" that are the effective parameters we want to discover.

One of the issues we will attempt to demonstrate through an \em{ad hoc} designed caricature, is that a
simple, generic parameter optimization problem, with a simple quadratic minimum and no parameter sloppiness
can -through a ``bad enough" transformation- be turned into a sloppy one.
%
This is not surprising, but it also suggests the converse: that sloppy problems are possibly only sloppy
in the given model parameter space, but could be transformed back into a different parameter space where
they would become ``nice" again.
%
One could try to find/approximate such a transformation ``back to nonsloppiness" - but then the new parameters
in this transformed space might have no obvious physical meaning or interpretation. 
%
More interestingly, one might try to uncover such a transformation through jointly data-mining parameter
space and model prediction space \em{with the right metric}; in our opinion, the effort to construct a meaningful
metric with which to navigate model-prediction space (and through it, possibly, also model parameter space)
is an important hallmark of the model reduction effort of Transtrum and coworkers.

When one discusses reduction of variables as well as parameters in physical models, the first technique
that arguably comes to mind is dimensional analysis - finding (through the Buckingham $\pi$ theorem)
nondimensional groupings of variables and parameters in terms of which the model can be expressed.
%
In this work we will not attempt to incorporate this last approach in our data-driven framework; we are
interested in detecting effective parameters, and possibly effective variables; we will not explore
methods that involve joint reduction in terms of mixtures of both. 


In presenting the theory, it is initially convenient to consider (a) choosing the number and nature of
measurements to be made and then (b) process \em{all possible} predictions for \em{all possible} 
parameter values together, globally identifying \em{all possible} limiting cases in which the
problem reduces (in possibly different ways). 
%
Several literature papers proceed in this form.
%
We are interested in computational navigating parameter space, taking into account the ``local"
sloppy directions: finding ways to usefully parametrized an appropriately transformed parameter
space so that it can be simply -and hopefully efficiently- searched by local Taylor series
searches (in the right effective directions). 
%
If successful, this would provide a ``parameter free" component of our equation-free/variable-free
framework: both state variables, parameters and the right hand sides of the relevant evolution equations
would be determined in a data-driven manner.
%
We also mention that we believe there is a strong connection (on the data-mining algorithm side) between
what we attempt to do here (the jointly-informed reduction of parameter space and state space) and the
recent developments in the geometry and analysis of dual networks on questionnaires \cite{A14}.


{\color{gray}
\paragraph{Sloppiness.}
Holy grail = modeling $\xrightarrow{data}$ calibration
$\xrightarrow{wrappers}$ understanding/prediction/aggregate
statistics/automated tasks (EF).
Field reality = parameter sloppiness:
compensatory nature of individually meaningful
(at detailed/fine level) parameters.
Parameter sloppiness in the literature:
Sethna, Transtrum, even Constantine.
Distinguish sloppiness from overparameterization.\footnote{`{\it The
origin of sloppiness is not a simple lack of data
where trivial overparametrization leads to unidentifiable parameters.}'
\cite{WCGBMBES06}}
Liken sloppiness to perturbation:
singularly (also regularly?) perturbed systems
are indifferent to changes in (fine) initialization
if macro-quantities are kept invariant.

\paragraph{Problem statement.}
Need non-sloppy effective parameters
to reduce complexity/effectuate tractability/explore system features
in parameter space.
Draw analogy to evolution laws in terms of effective state variables.
Carry analogy further:
detailed models transparent (few simple rules) but large,
coarse-grained models insightful but non-trivial to derive;
individual parameters easily interpretable but sloppy,
effective parameters non-sloppy but non-trivial to derive and interpret.
Re-iterate problem statement.

\paragraph{Problem solution.}
Suggested partial solution: data-driven identification of effective parameters.
[Yannis. Include literature.]

\paragraph{Peripherals.}
Several considerations arise in the process:
some new, some recast in a new light.
(a) Helpful/insightful transformations of parameters.
(b) Duality (of some sort) between state-/parameter space.
(c) Contrasting singular/regular perturbations and their phenotypes.

\paragraph{Extensions.}
[Yannis.]
}



\section{Meaning and origin of sloppiness}

\subsection{Setting}
%
We frame our discussion in the context of multivariable vector functions $x(t;\p) \in \R^d$.
The dimension $d$ is problem-specific and thus arbitrary.
The set of independent variables has been partitioned in two parts, the bona fide variables $t = (t_1,\ldots,t_K) \in I \subset \R^K$ and the parameters $\p = (\p_1,\ldots,\p_M) \in \ps \subset \R^M$; the dimensions $K$ and $M$ are likewise arbitrary.
In this manuscript, we will restrict $x$ to be the solution to an initial/boundary value problem for a set of differential equations posed over some domain $I$.
As such, $x$ will typically be known only implicitly.
In such a setting, the independent variables are split into $t$'s and $\p$'s according to whether derivatives of $x$ with respect to them appear in the differential equations or not.
In all our examples below,  $t\in\R$ will be interpreted as time and $x$ will satisfy an ODE system, whereas $\p$ will enter the problem parametrically.\\

For each fixed $\p\in\ps$, the \emph{full model response} $\{x(t;\p) \}_{t \in I}$ is an element of some \emph{full model space} $\fms$, typically chosen as an appropriate function space.
We denote by $\fmr : \ps \to \fms$ the map sending specific parameter values to
full model responses, i.e. $\fmr(\p) = \{ x(t;\p) \}_{t \in I} \in \fms$.
As $\p$ ranges over all possible values in $\ps$, the response $\fmr(\p)$ traces out the (generically $M-$dimensional) \emph{full model manifold} $\fmm =
%\mathrm{graph}(\fmr)
%=
\{ \fmr(\p) \}_{\p \in \ps}
\subset
\fms$
which is  generically $M-$dimensional.
We will assume that manifold to be $C^1$, i.e. $x(t;\p)$ to be continuously differentiable with respect to parameters.\\

One is only rarely interested in observing the \emph{full} model response,
which amounts to monitoring the system state for \emph{all} $t \in I$.
Instead, one observes one or more \emph{functionals} of that response, say $f^*_1 , \ldots , f^*_N : \fms \to \R$.
For the purposes of our discussion here,
these functionals can be thought of as \emph{constraining} the solution
in some way---e.g., as returning the value $x_j(t_i)$
of a state vector component at some specific time.
For any $\p\in\ps$, we term each individual $f^*_n(\fmr(\p))$
a \emph{partial observation} of the system and the $N-$tuple
%
\[
 \mu(\p)
=
 f^*(\fmr(\p))
=
\left[\begin{array}{c}
 f^*_1(\fmr(\p)) \\ \vdots \\ f^*_N(\fmr(\p))
\end{array}\right]
\in
 \R^N
\]
%
an \emph{observed model response} or, when confusion cannot arise, \emph{model response}.
Under the action of $f^*$,
the full model manifold $\fmm$ is projected to
the \emph{(observed) model manifold} $\omm = \{ \omr(\p) \}_{\p \in \ps}$
in the \emph{(observed) model space} $\R^N$.
For any fixed observation $z = (z_1,\ldots,z_N) \in \omm$ and $1 \le n \le N$,
the individual fiber $\fiber_{n,z_n} = {f^*_n}^{-1}(z_n)$
is of \emph{co-dimension one} in $\fms$
and thus intersects the full model manifold $\fmm$ (generically) along
an $(M-1)$-dimensional submanifold $\fmm_{n,z_n} = \fiber_{n,z_n} \cap \fmm$.
%for linear functionals $f^*_n$,
%it is an affine copy of $\mathrm{Ker}(f^*_n)$.
This submanifold corresponds, in turn, to a region
$\ps_{n,z_n} = \fmr^{-1}(\fmm_{n,z_n})$ of the same dimension
in parameter space $\ps$.
\emph{All parameter combinations in that region yield model responses
consistent with the given partial observation $z_n$
in that $(f^*_n\circ\fmr)(\ps_{n,z_n}) = z_n$.}
Increasing the number of fixed partial observations one at a time,
one decreases the dimension of the parameter space region consistent with them until,
at $N=M$ and further, that becomes zero.
This is evident in that $\D_\p \omr$ is of rank $M$
and $\omr : \ps \to \omm$ is one-to-one.
We will assume $\omr$ to be a homeomorphism between $\ps$ and $\omm$,
which also makes it an atlas for the latter.

\subsection{Parameter sensitivity}
%
To discuss sensitivity to parameter variations,
one must be able to assess the \emph{magnitude} of shifts in observations
relative to such variations.
Here also, as in \cite{TMS11}, we measure distances on $\omm$
using a Riemannian metric effectively induced
by the norm in the carrier space $\R^N$.
We take that norm to be the standard Euclidean norm $\norm{\cdot}$,
sidestepping in the process many important questions
relating to weighing of partial observations.
In that setting, an infinitesimal displacement
$d\p = (d\p_1 , \ldots , d\p_M)^{\rm T}$ in $\ps$
yields the infinitesimal displacement
$dz = (dz_1 , \ldots , dz_N)^{\rm T} = (\D_\p\mu) d\p \in \mathrm{T}_\p\omm$
with length $\norm{dz}^2 = (d\p)^{\rm T} \, g \, d\p$.
The $M \times M$ positive definite matrix $g = (\D_\p\omr)^{\rm T}(\D_\p\omr)$
is the \emph{metric tensor} for $\omm$
for the specific atlas $\omr$.\\

Since $\mathrm{rank}(\D_\p \omr) = M$ by assumption,
both the \emph{full} and the \emph{observed} system
are responsive to \emph{all} parameter variations.
Locally around some point $\p \in \ps$, though,
the observed model response may vary greatly with the direction of $d\p$
and be disproportionately small along certain directions.
This phenomenon is termed \emph{sloppiness},
and it demonstrates itself in the spectrum of the metric.
Small eigenvalues yield small observed responses,
with sloppy directions in $\ps$ being the pull-backs under $\D_\p \omr$
of the associated eigendirections in $\mathrm{T}_\p\omm$.

\subsection{Origins of sloppiness}
%
One of the ways in which sloppiness arises
is through observation functionals
that are \emph{nearly} linearly dependent. 
Indeed, linear dependencies among functionals
reduce the dimensionality of $\omm$,
so a \emph{continuous deformation} of a linearly dependent such set
fleshes out a manifold with a hyper-ribbon structure.
Pictorially, this is a process similar to inflating a mattress.\\

To identify other factors contributing to sloppiness,
we investigate how the projections $f^*_1,\ldots,f^*_n$
transform the metric $h = (\D_\p\fmr)^{\rm T}(\D_\p\fmr)$ on $\fmm$
to the metric $g = (\D_\p\omr)^{\rm T}(\D_\p\omr)$ on $\omm$.
For clarity of presentation,
we restrict our attention to the case where
$f^*_1,\ldots,f^*_N$ are linear and linearly independent,
whereas $\fms$ is a Hilbert space
with inner product $\langle\cdot,\cdot\rangle_\fms$
and induced norm $\norm{\cdot}_\fms$.
In that setting,
$\fms$ is isomorphic to its dual $\fms^*$,
so there exist $f_1,\ldots,f_N \in \fms$
satisfying $f^*_n = \langle f_n,\cdot \rangle_\fms$.
It follows that
%
\[
 \omr(\p)
=
\left[\begin{array}{c}
 \langle f_1,\fmr(\p) \rangle_\fms
\\
\vdots
\\
 \langle f_N,\fmr(\p) \rangle_\fms
\end{array}\right] ,
\ \mbox{with push-forward} \
 \D_\p\omr\vert_\p
=
\left[\begin{array}{c}
 \langle f_1,\cdot\rangle_\fms
\\
\vdots
\\
 \langle f_N,\cdot \rangle_\fms
\end{array}\right]
 \D_\p\fmr\vert_\p .
\]
%
The metric on $\omm$ becomes
%
\[
 g\vert_{\omr(\p)}
=
 (\D_\p\omr\vert_\p)^{\rm T}
 (\D_\p\omr\vert_\p)
=
 (\D_\p\fmr\vert_\p)^{\rm T}
 f f^*
 (\D_\p\fmr\vert_\p) ,
\] 
%
with $f f^* : \fms \to \fms^*$ an operator
that we schematically write as
%
\[
\begin{array}{c}
 f f^*
=
\\
\vspace*{4mm}
\end{array}
\begin{array}{c}
\Big[
 \langle f_1,\cdot\rangle_\fms
,
 \ldots
,
 \langle f_N,\cdot\rangle_\fms
\Big]
\\
\vspace*{4mm}
\end{array}
%
\left[\begin{array}{c}
 \langle f_1,\cdot\rangle_\fms
\\
 \vdots
\\
 \langle f_N,\cdot\rangle_\fms
\end{array}\right] .
\]
%
This relation connects $g$ to $h$
through the mediation of $f f^*$.\\

To understand the effect of that mediation,
we interpret $f f^*$ as a symmetric endomorphism on $\fms$
by means of the isomorphism between $\fms$ and $\fms^*$.
Its spectrum consists of the zero eigenvalue,
linked to the co-dimension $N$ kernel
$\mathrm{Ker}(f f^*) = f_1^\perp \cap \ldots \cap f_N^\perp$,
and of a nontrivial part linked to the invariant subspace
$\mathrm{Im}(f f^*) = \mathrm{span}(f_1,\ldots,f_N)$.
It is trivial to show that
%
\[
 f f^*\vert_{\mathrm{Im}(f f^*)}
=
 f^* f
=
\left[\begin{array}{ccc}
 \langle f_1,f_1\rangle_\fms
&
 \ldots
&
 \langle f_1,f_N\rangle_\fms
\\
 \vdots
&
 \vdots
&
 \vdots
\\
 \langle f_N,f_1\rangle_\fms
&
 \ldots
&
 \langle f_N,f_N\rangle_\fms
\end{array}\right] ,
\]
%
which proves the nontrivial part of the spectum to only contain
the eigenvalues of the $N \times N$ Gram matrix $f^* f$.
The columns
$\D_{\p_1}\fmr\vert_\p , \ldots , \D_{\p_N}\fmr\vert_\p
\in \mathrm{T}_{\fmr(\p)}\fmm \simeq \fmm$
of $\D_\p\fmr\vert_\p$ can be decomposed into components along
$\mathrm{Ker}(f f^*)$ and $\mathrm{Im}(f f^*)$,
%
\[
 \D_{\p_n}\fmr\vert_\p
=
 \mathrm{N}_n(\p)
+
 \left[ f_1 , \ldots , f_N \right]
 \mathrm{I}_n(\p) ,
\quad \mbox{with} \
 \mathrm{N}_n(\p) \in \mathrm{Ker}(f f^*) .
\]
%
Collecting these decompositions, then, we find
%
\[
 \D_\p\fmr\vert_\p
=
 \mathrm{N}(\p)
+
 \left[ f_1 , \ldots , f_N \right]
 \mathrm{I}(\p)
\]
%
and thus further obtain
%
\be
 g\vert_{\omr(\p)}
=
 \mathrm{I}(\p)^{\rm T}
\left(
 f^* f
\right)^2
 \mathrm{I}(\p) .
\label{g-ito-proj}
\ee
%
This equation expresses the metric on $\omm$
in terms of the Gram matrix of the projections
and of the projection of $\mathrm{T}_{\fmr(\p)}\fmm$ on their image.\\

It now becomes evident that sloppiness may arise
because either $f^* f$ or $\mathrm{I}(\p)$ are ill-conditioned.
The former of these matrices depends on
the pairwise norms and angles of $f_1,\ldots,f_N$,
and its spectrum characterizes both their relative weight
and the degree to which they are linearly independent.
Sloppiness, in this context, is due to functionals
that are poorly scaled (disparate norms)
or nearly linearly dependent (disparate angles).
This was the case in \cite{WCGBMBES06},
where sloppiness was mediated by a Vandermonde matrix
specific to Taylor polynomials.
Replacing those with any orthonormal polynomial basis
would have presumably sufficed to remove sloppiness.
If $\mathrm{I}(\p)$ is ill-conditioned, instead,
then sloppiness is due to directions in $\mathrm{T}_\p\fmm$
that are nearly parallel to the projection kernel
$\mathrm{Ker}(f^*)
=
\mathrm{Ker}(f^*_1) \cap \ldots \cap \mathrm{Ker}(f^*_N)$.
Here, although the observation functionals are well-chosen
and the \emph{full} model response to parameter variations is appreciable,
the latter is projected to a much smaller \emph{observed} model response.
A prime such example is offered by multiscale systems.
As we shall see shortly,
many parameter directions in such systems
only affect behavior at unobserved scales,
effectively identifying directions in $\fmm$
irrelevant to the monitoring protocol.\\

{\color{gray}\paragraph{Caricature model space.}
Distinguish output from cost function: former fixes model (=data)
space, latter space norm (i.e., packages data into a scalar).
\emph{Output} analogous to random variable: constrains detailed system
output (i.e., detailed state at all times).  In effect, output formed
by projecting detailed system output (=a point in some space) to ($N$
in number) $1-$D subspaces $\rightarrow$ observables are projections
(along co-dim. $N$ kernel intersection) of system output onto an
$N-$dim subspace [nonlinear kernel fibrations OK].  Model manifold
constructed as projection of `detailed model manifold.'  Norm in model
space $\rightarrow$ semi-norm in `detailed model space': balls
$\rightarrow$ cylinders (ball $\times$ kernel) $\rightarrow$
``neutral" directions along projection kernel.  Sloppy directions in
parameter space are mapped to neutral (i.e.kernel) directions on
`detailed model manifold', i.e. $d\theta$ is (largely) along
intersection of kernels $\rightarrow$ maps to much thinner
displacement on (restricted) model space.  Conclusion: lower-dim
subsets of parameter space may map to directions along kernel, but
these do not necessarily extend through entire parameter space:}
discuss two cases: (a) sing. pert. caricature with $\eps\in\R$ (only
$1/\eps$ above threshold maps to kernel directions, whereas $1/\eps$
below it---incl. negative values---does not: detailed model manifold
is 'sigmoidal') and (b) Goldbeter--Koshland modules, where steady
states (observables) show `sloppiness--sharp transition--sloppiness'
structure.  Eye of the beholder again: metric in parameter space bad
predictor of (constrained) model response.



\section{Introductory examples}
%
In the last section, sloppiness was introduced as a \emph{local} property
pertaining to the specific combination of model and observables,
a fact reflected in our analysis through the use of linear information (Jacobians, tangent spaces).
However, sloppy behavior is an \emph{open property}
characterizing entire regions of parameter space.
This fact has strong implications,
the foremost of which is that sloppy system behavior
is affected by a \emph{reduced} number of effective parameters.
In this section, we elucidate these ideas through a list of elementary examples.
Then, in the remainder of this paper, we develop a data mining framework
for the \emph{identification} of such parameters.\\

The footprint of sloppiness in parameter space
is the existence of extended, effectively lower-dimensional regions,
all points of which evoke \emph{nearly identical} observed model responses.
Effective system parameters are roughly constant on each such ``effective surface'',
so mapping out the latter amounts to identifying the former.
In the linear picture we painted above,
an $M-$dimensional ball in $\mathrm{T}_{\omr(\p)}\omm$
is pulled back to an ellipsoid centered at $\p$
with widely disparate principal axes.
In that setting, principal directions correspond to sloppy parameter combinations
and can be unraveled by Linear Algebra techniques
such as Principal Component Analysis (PCA), cf.~\cite{ADS06}.
In reality, pronounced deviations of the observed model manifold from linearity
carry points on $\omm$ within a specific distance from $\omr(\p)$
to a curved region in parameter space
resembling a lower-dimensional surface.
We shall call such regions \emph{neutral sets} and, directly below,
build a \emph{nonlinear} data mining computational framework
to identify them and the associated effective parameters.\\

\subsection{An elementary example \label{ss-elem.ex}}
%
To illustrate the concepts introduced above,
we begin with the prototypical nonlinear dynamical system
%
\be
\begin{array}{rcl}
 \dot{x} &=& y - \lambda x ,
\vspace*{1mm}\\
 \eps \dot{y} &=& \eps x - \displaystyle\left(1+\frac{10}{1.5-\sin y}\right) y ,
\end{array}
\ \mbox{supplemented with} \
\begin{array}{rcl}
 x(0) &=& x_0 ,
\vspace*{1mm}\\
 y(0) &=& y_0 .
\end{array}
\label{elem-ODE}
\ee
%
Here, dots denote differentiation with respect to time $t \in I = [0,\infty) \subset \R$, so $K=1$.
This model has a unique, globally attracting steady state at the origin
whose stability specifics are controlled by $\eps$ and $\lambda$.
All four of $(\eps,\lambda,x_0,y_0)$ can be viewed as parameters,
but we reduce their number to $M=2$ by setting $\p = (\eps,y_0)$ and fixing $\lambda= 2$ and $x_0 = 1$.
The full model response is the full system trajectory $\{(x(t;\p),y(t;\p))\}_{t \in I}$, so that $d = 2$ and $\fms = C^1(0,\infty) \times C^1(0,\infty)$ is infinite-dimensional.
For our observed model response, we choose
%
\be
 \mu(\p) = \big( y(t_1;\p) \,,\, y(t_2;\p) \,,\, y(t_3;\p) \big) ,
\ \mbox{for fixed time instants} \
 0 < t_1 < t_2 < t_3 .
\label{elem-mu}
\ee
%
It follows that the $2-$D observed model manifold is embedded in $3-$D Euclidean space ($N=3$).
A segment of this highly nonlinear manifold is plotted in Fig.~\ref{f.elem.ex.1}.\\

As emphasized earlier, different regions of $\omm$ may or may not be sloppy.
For model~\eqref{elem-ODE} with the given observables, sloppiness arises in the regime $\eps \ll 1$ where the system is \emph{singularly perturbed}.
In that regime, and during an initial fast transient, all trajectories enter an $\Ord(\eps)$ neighborhood of the $x-$axis containing an attracting, global slow manifold.
Trajectories subsequently remain exponentially close to that manifold and each other, so that their fast component $y$ remains $\Ord(\eps)$.
In that manner, information on each trajectory's \emph{initial} fast component $y_0$ is effectively \emph{erased} during the transient.\\

This last observation is crucial in understanding how sloppiness arises in a multiscale setting.
As $\eps \downarrow 0$, the slow phase sets on earlier and $t_3 > t_2 > t_1$ fall in it one after another.
When this happens for e.g. $t_3$, the corresponding partial observation $y(t_3)$ approaches zero irrespectively of the $y_0-$value that generated it: $y_0$ becomes sloppy.
This process is evident in the right panel of Fig.~\ref{f.elem.ex.1}: as $\eps$ decreases, the outputs generated by \emph{all} initial fast components (within a bounded interval) are seen to collapse to the origin.
Plainly, this situation generalizes easily: for a general multiscale system possessing an attracting slow manifold and a fast fibration, all points on a given fast fiber will generate similar observations because they possess matching outer solutions.\\

The hallmark of sloppiness is the existence of neutral sets in parameter space.
Figure~\ref{f.elem.ex.2} shows (a portion of) the same observed model manifold $\omm$, together with a point $p^* = \omr(\p^*) \in \omm$ in the sloppy regime and a $\delta-$neighborhood (ball) $\mathrm{B}_\delta(p^*)$ around it.
The ball encloses the submanifold $\omm \cap \mathrm{B}_\delta(p^*)$, all points of which are within a distance $\delta$ from $p^*$.
The inverse images of these points under $\omr^{-1}$ (shown in the right panel; to come) constitute the \emph{neutral set at level $\delta$}, denoted by $\ps_\delta(\p^*)$, all parameter values in which yield observed model responses $\delta-$close to $p^*$.
Note carefully that this set is specific to our choice of model space metric, since that choice affects $\mathrm{B}_\delta(p^*)$ strongly.
Plainly, $\ps_0(\p^*) = \{\p^*\}$ as there are no indeterminacies in this setting; different parameter sets yield different responses.
As $\delta$ grows, $\ps_\delta(\p^*)$ becomes a growing elliptic disk.
Smaller $\eps-$values can accommodate wider intervals of $y_0$ than large ones within the same threshold $\delta$, because the fast contraction is larger; this results in the left--right asymmetry evident in the same figure (to come).
For $\delta$ past a certain threshold, $\ps_\delta(\p^*)$ opens up and becomes unbounded.
Evidently, that threshold is reached when the ball $\mathrm{B}_\delta(p^*)$ reaches the $\eps=0$ boundary of $\omm$.\\

In light of this observation it becomes clear that, the closer $\p^*$ is to the $\eps = 0$ boundary of $\omm$ to start with, the smaller the tolerance $\delta$ for which $\ps_\delta(\p^*)$ becomes unbounded.
This is evident in Fig.~3 (do we want such a figure?), where we have plotted the neutral sets $\ps_\delta(\p^*)$ corresponding to various points $\p^*$ and tolerances $\delta$.
In the singularly perturbed regime, then, the system is effectively \emph{parameter-free}: all parameter combinations in a wide set yield responses within tolerance.\\

THINGS TO DO:\\
(1) Color left panel of first figure by $1/\eps$ (not $\eps$); extend range of $y_0$; add color bar.\\
(2) Chop wing tail in second figure.\\
(3) Add panel to second figure with contours. For that, we'll need Alexander's pseudo-arc length continuation algorithm.\\
(4) Add two-panel, third figure with contours in parameter space for two $p^*$ (differing only in $\eps$, not in $y_0$) and various $\delta$.

%Interestingly, the dimensionality of the neutral set in this system
%changes depending on the parameter $\epsilon$.
%This is represented by Fig. \ref{fig:sing-pert:x20}.
%In this regime, the neutral set is two-dimensional: neither $\eps$ nor
%$x^2_0$ affect the observed model response, i.e. $x^2$ at $t_1$, $t_2$ and $t_3$.
%However, as $\eps$ increases, we approach a state in which predictions are nearly
%constant for different $x^2_0$, but further changes in $\eps$ does in fact affect $\mu$ as we begin sampling off the slow manifold.
%In this case, we have a one-dimensional neutral set: $\eps$.
%Finally, when $\eps \approx O(1)$, both parameters significantly change $\mu$ and our
%neutral set is empty.
%Fig.~\ref{fig:sing-pert-slopp} captures each of these three regimes.


%
\begin{figure}[t]
\scalebox{0.24}[0.24]{
\includegraphics{clrd_by_params_no2.pdf}
}
\caption{\label{f.elem.ex.1}
The observed model manifold $\omm$ for system~\eqref{elem-ODE} in $3-$D \emph{observed model space}. The two parameters here are $\epsilon$ and $y_0$, and the map $\omr$ from 
parameter to (observed) model space is given in~\eqref{elem-mu}. The manifold has been colored by each parameter to visualize how these vary on it. The manifold bottom corresponds to the singularly perturbed regime $0 < \eps \ll 1$, see left panel. In that regime, widely different initial conditions yield nearly identical model responses, see right panel.}
\end{figure}
%

%
\begin{figure}[t]
\scalebox{0.24}[0.12]{
\includegraphics{clrd_by_eps.pdf}
}
\caption{\label{f.elem.ex.2}
Same manifold as in Fig.~\eqref{f.elem.ex.1} seen from a different angle. The green submanifold contains all points within distance $\delta$ from a point $\p^* \in \omm$ (black dot), and it is the intersection of $\M$ with the $\delta-$ball centered at $\p^*$ (light blue ball).}
\end{figure}
%

{\color{gray}\paragraph{Caricature $1 \rightarrow 0$.}
Setting: planar singularly perturbed system + data in sloppy $\eps-$regime.
Limit $\eps \downarrow 0$ entirely removes parameter
without sacrificing output accuracy.
Outcome: parameter-free system.}

\paragraph{Caricature cost function.}
{\color{gray}Discuss how cost function determines neutral set.}
Include short discussion on a priori knowledge of constants
possibly lifting sloppiness (cite fwd to MM).
Discuss free vs constrained (e.g. log-likelihood) choice of cost function.
Cost function choice should be motivated
partly by interest (e.g. which components to include and how)
and partly by statistics/data fitting concerns
(e.g. log-likelihood $\rightarrow$ quadratic sum).


\subsection{Sloppiness in the eye of the beholder}

\paragraph{Overall plan.}
Same setting.  Map $\eps \mapsto 1/\eps$ maps interval
$[0,\eps_{\rm max}]$ to ray $\rightarrow$ sloppiness not removable
through linear rescaling.  [Here again, transformations.]  Carry on:
flat neutral set $\mapsto$ spiral/Swiss roll.  Outcome: sloppiness can
be more than a finite stretching even in the absence of singular
perturbations.\\

Contrary to parameter identifiability which is a binary notion (a
parameter is either identifiable or not), sloppiness is
\emph{relative} and affected by various factors, the choice of model
parameters and associated parameter scales featuring prominently among
them.  This is already evident in the model of
Section~\ref{ss-elem.ex}, where we found that $1/\eps$ was sloppy.
There, for fixed initial conditions and any preset threshold
$\delta > 0$, there exists $\eps_\delta \ll 1$ such that all model
responses generated by any $\eps \in (0,\eps_\delta)$ are
$\delta-$close to each other.  In terms of $1/\eps$, the neutral set
is the unbounded interval $\ps_\delta = (1/\eps_\delta,\infty)$.  In
terms of $\eps$, however, that interval is radically shrunk to
$[0,\eps_\delta)$.  This demonstrates that sloppiness can be
intrinsically nonlinear and is not solely the result of poor scaling
choices; here, the unboundedness of $\ps_\delta$ can only be removed
by a bona fide nonlinear transformation such as $\eps \mapsto 1/\eps$
above.\\

We stress this point further by means of an elucidating (if contrived) example.
Starting from the linear, singularly perturbed, analytically tractable ODE system
%
\be
\begin{array}{rcl}
 X &=& -\lambda X ,
\\
 \eps Y &=& - Y ,
\end{array}
\quad\mbox{with}\ \eps \ll 1 ,
\label{XY-system}
\ee
%
we construct a nonlinear one through the change of variables
%
\[
 X = x - \beta y^2
\quad\mbox{and}\quad
 Y = y - \alpha x^2 .
%\label{xy-vs-XY}
\]
%
Here, $\alpha$ and $\beta$ are additional parameters.  This
re-coordinatization of the phase plane maps the global slow manifold
to $y=\alpha x^2$ and the family of fast fibers to
$\{x=\beta y^2 +c \ \vert \ c\in \R\}$; the corresponding timescales are
$1/\lambda$ and $1/\eps$.  In all simulations below, we fix
$\eps = 10^{-3}$, and $\beta = 10^{-2}$ with initial conditions
$x(0) = y(0) = 1$. We define the model response to be

\begin{align}
  \omr(\alpha, \lambda) = \begin{pmatrix} x(t_1) & x(t_2)
    & \hdots &  x(t_{10}) \\ 
    y(t_1) & y(t_2) & \hdots & y(t_{10}) \end{pmatrix} \; \in  \; \mathbb{R}^{2 \times 10} \\
  \label{eq:henon-mr}
\end{align}

where times are evenly spaced on the interval $[0.1, 1.0]$. In this context, the remaining parameters
$\alpha$ and $\lambda$ are not sloppy.
When poorly transformed, however, they can exhibit highly nonlinear sloppiness.\\

To starkly demonstrate this, we define new parameters $(\p_1,\p_2) =
(H \circ H)(\alpha,\lambda)$, with $H : \R^2 \to \R^2$ the H\'enon map
tuned to its chaotic regime. That is,

\begin{align*}
  H(\alpha, \lambda) = \begin{bmatrix} 1 - a \alpha^2 + \lambda \\  b
    \alpha \end{bmatrix}
\end{align*}

and $a = 1.3$, $b = 0.3$. In terms of these transformed variables
our model becomes

\begin{align}
  \begin{bmatrix} \dot{x} \\ \dot{y} \end{bmatrix}  = \frac{1}{1 -
  4 \bigg( \frac{\p_1}{b} - 1 + a \bigg( \frac{k}{b} \bigg)^2 \bigg) \beta x y} \begin{bmatrix} 1 &
  2\beta y \\ 2 \bigg( \frac{\p_1}{b} - 1 + a \bigg( \frac{k}{b} \bigg)^2 \bigg) x  &
  1 \end{bmatrix} \begin{bmatrix} - \frac{k}{b}(x - \beta y^2) \\ -(y -
  \p_1)/\epsilon \end{bmatrix} 
\label{m.henon}
\end{align}

with $k = {\p_2 - 1 + a
  \big(\frac{\p_1}{b}\big)^2}$. Following the procedure outlined in the last
section, we then fix $(\alpha^*,\lambda^*) = (1,1)$, compute the
corresponding $(\p_1^*,\p_2^*) = (H \circ H)(\alpha^*,\lambda^*)$ and
mine parameter values inside the neutral set at the preset level
$\delta = 2.0$. \\

To demonstrate how a collection of sloppy parameter combinations might
arise in a typical setting, we mine parameter values not through some
sampling procedure as will be the case in later examples, but by
repeatedly fitting our model to the base response $\omr(\p_1^*,
\p_2^*) = \omr^*$. We use a standard least squares objective

\begin{align}
  c(\p_1, \p_2) = \| \omr(\p_1, \p_2) - \omr^* \|_2^2
\end{align}

In this context, $\delta$ refers to the tolerance of our optimization
routine: we keep any point for which $c(\p_1, \p_2) < \delta$. By
initializing this routine at different values of $\p_1$ and $\p_2$, we
converge to different points in parameter space, eventually obtaining
a collection of parameter combinations that all fit the data within
our tolerance.  Figure~\ref{f.transf-params} shows this collection:
Figure~\ref{f.henon} corresponds to the actual output of our model
fitting, while Figure~\ref{f.henon-inverse} reveals what this
collection looks like in terms of the original parameters $\alpha$ and
$\lambda$ . Plainly, while the latter is not sloppy as expected, the
former is, as the characteristic length scales differ by no less than
two orders of magnitude. Thus, through a proper nonlinear
transformation, we can remove the sloppiness in our model parameters.

%
\begin{figure}[ht!]
  \begin{subfigure}[t]{0.49\textwidth}
    \centering
    \includegraphics[width=\textwidth]{transformed-params-fromoptimization-insert}
    \subcaption{Parameter values found through repeated fitting of
      Model \ref{m.henon}. \label{f.henon}}
  \end{subfigure}
  \begin{subfigure}[t]{0.49\textwidth}
    \centering
    \includegraphics[width=\textwidth]{inverted-params}
    \subcaption{Original parameter values found by inverting the
      collection in the left panel.  \label{f.henon-inverse}}
  \end{subfigure} %
  \caption{The set $\ps_\delta$ corresponding to the point
    $(\alpha_*,\lambda_*) = (1,1)$ with model response $\mu$ as in
    Eq.~\ref{eq:henon-mr} and $\delta = 2.0$. The absence of
    sloppiness is evident in the right panel, in which $\ps_\delta$ is
    plotted in terms of the original parameter set
    $(\alpha,\lambda)$. To the contrary, the same domain plotted in
    terms of the transformed parameters $(\p_1,\p_2)$ appears bent and
    sloppy. \label{f.transf-params}}
\end{figure}


\subsection{Alexander's model}

\paragraph{Overall plan.}
$3 \rightarrow 1$ parameters. DMAPS
$\rightarrow$ sloppy set $\rightarrow$ effective parameter. Outcome:
nonlinear data mining + transversal propagation $\rightarrow$
effective parameters.  Also discuss non-uniqueness of effective
parameters (anything transversal); candidate parameter combinations
are tested through non-degeneracy of Jacobian.\\

The simple, singularly perturbed system presented in
Sec. \ref{ss-elem.ex} contained up
to two neutral directions in parameter space: $\eps$ and $y(0)$,
which created the ellipsoidal contours seen in
Fig. \ref{fig:sing-pert:contour-2}. The fact that the bare parameters
themselves were sloppy, and not some hidden nonlinear combination
of them, makes it an especially simple example. Next we consider the
more general case in which model sloppiness arises in parameter
combinations. \\

The model in question is the chemical reaction system

\begin{align*}
  A \xrightleftharpoons[k_{-1}]{k_1} B \xrightarrow[]{k_2} C .
\end{align*}

Under the assumption that species $B$ is in quasi equilibrium, valid
when $k_2 \gg k_{-1}$, the simplified solution for the product
concentration is given by

\begin{align*}
  C_C = C_{A_0}(1 - e^{-k_{eff} t}) \quad \mathrm{with} \quad k_{eff} = \frac{k_1 k_2}{k_{-1} + k_2}
\end{align*}

Thus, when operating in this quasi-steady state regime, although our
base model includes the three bare parameters $k_1$, $k_{-1}$ and
$k_2$, the resulting dynamics are governed by a single effective
parameter $k_{eff}$. This implies that there will be an entire
two-dimensional surface in parameter space, each point on which
equally well fits experimental data. In addition, this neutral set
will be highly nonlinear, as it is effectively a level set of the
nonlinear equation $k_{eff} = \frac{k_1 k_2}{k_{-1} + k_2}$. More
explicitly, any combination of the bare parameters $k_1$, $k_{-1}$ and
$k_2$ that yield the same value for $k_{eff}$ will predict the same
system dynamics. Fig. \ref{fig:qssa:sloppy-manifold} shows a surface
in parameter space along which $k_{eff} = 1$.

\begin{figure}[htbp]
  \centering
  \includegraphics[width=\linewidth]{keff-1}
  \caption{The surface in parameter space over which $k_{eff} = 1$. \label{fig:qssa:sloppy-manifold}}
\end{figure}


It is important to note that this surface in parameter space captures
all the sloppiness hidden in our model. Traversing along the manifold we
find many different parameter combinations with nearly equivalent
model predictions; moving in a direction transverse to the surface
brings us to a new $k_{eff}$ and thus significantly different model
predictions. Unfortunately, real-life systems will typically contain
tens to hundreds of parameters, preventing us from simply plotting the
neutral set and visually inspecting its sloppiness. Instead, this
sloppy surface will be hidden somewhere within the high dimensional
parameter space. Our objective then becomes to uncover the neutral set
in a potentially high-dimensional parameter space and to develop a
parameterization of its potentially nonlinear structure. \\

% In fact, we must first confront a separate challenge: given some
% model, how do we generate its underlying sloppy surface? One could
% simply sample parameter space and keep those points for which the
% objective function fell within some tolerance, but this becomes
% expensive and inefficient in high dimensions. Instead, we achieve this
% aim by repeatedly fitting our model starting from different initial
% guesses in parameter space. When dealing with an ideal model, each
% initial guess would converge to the same, global optimum in parameter
% space. However, sloppiness creates a manifold of solutions, each of
% which yield equally good fits. Thus, by starting our optimization
% routine at different points in parameter space, we will naturally
% converge to different values on the sloppy manifold, eventually
% generating a dense cover as shown in Fig. \ref{fig:qssa:opt}.

First, we must confront the challenge of generating a collection of
sloppy parameters. At present, we fix our model response $\omr^*$ and
implicitly the corresponding parameters $\theta^*$, and then
investigate the set of parameters that map to a model response within
distance $\delta$ of $\omr^*$ by sampling around $\theta^*$. That is,
by sampling parameter space we find points in the set
$S = \{\theta \ssep  \| \omr(\theta) - \omr^* \| < \delta \}$. \\

Given this dataset, we turn to the problem of uncovering its hidden structure. This
problem is naturally framed as one of dimensionality reduction: how
can we more efficiently describe our high-dimensional dataset? One
might initially consider performing a principal component analysis;
however, as Fig. \ref{fig:qssa:sloppy-manifold} shows, our manifold is
highly nonlinear. Instead, we turn to Diffusion Maps, a data mining
technique capable of uncovering such nonlinear structure. \\

\subsubsection{Diffusion Maps (DMAPS)}



 % description of DMAPS here? %

When we apply DMAPS to the set of parameter combinations found through
repeated optimization, we indeed uncover a parameterization of the
nonlinear system sloppiness. In this case, DMAPS provides a
two-dimensional embedding of the surface, as shown in
Fig. \ref{fig:qssa:dmaps}. \\

However, it would also be useful if we could uncover not only a
parameterization of the sloppy directions in our model, but also the
important parameter directions.

% alternative dmaps description %



\section{The case for data mining sloppy systems}

\paragraph{motivation.}
(re-)introduce neutral set in parameter space.  Motivate data mining
of neutral set for effective parameter identification.  Data mining
via PCA done \cite{ADS06} but neutral sets extended so nonlinearities
pronounced $\rightarrow$ nonlinear data mining such as diffusion
maps.

Alexander's Model with plain vanilla and mixed kernel.

\section{Parameters and variables}

\paragraph{Variables as parameters.}
Initial conditions subject to inference $\rightarrow$ link between
parameter space and state space.  Entire fast fiber sloppy.  Fast/slow
are relative notions, fixed by observer's (measurement) timescale,
i.e. by time grid (discrete case) or by initial relaxation window
(continuous case).

\paragraph{Parameters as variables.}
Optimization dynamics in data fitting $\rightarrow$ dynamics in
parameter space. State space dynamics + choice of cost function shape
the landscape in parameter space that the optimization algorithm
navigates.  Relatedly: state space dynamics $\rightarrow$ model
manifold; cost function $\rightarrow$ model space norm.

\paragraph{Parameters as variables.}
Joint parameter-/variable-space.
[What can we contribute here?]


\section{Sloppiness in enzyme kinetics}




\section{Metrics}


\section{Discussion}

Discuss model space and state space.\\

Discuss extension by following manifolds.\\

Discuss bounded/unbounded spaces.\\


\begin{thebibliography}{99}
%

\bibitem{AS72}
M.~Abramowitz and I.~A.~Stegun,
Handbook of Mathematical Functions
with Formulas, Graphs, and Mathematical Tables (10th ed.),
National Bureau of Standards, Washington, DC, 1972.

\bibitem{ADS06}
P.~Achard and E.~De~Schutter,
Complex parameter landscape for a complex neuron model,
{\it PLoS Comp. Biol.} \textbf{2(7)} 0794--0804, 2006.

\bibitem{A14}
J.~I.~Ankenman,
Geometry and Analysis of Dual Networks on Questionnaires,
Ph.D. Thesis, Yale University, 2014.

\bibitem{JG11}
K.~A.~Johnson and R.~S.~Goody,
The original Michaelis constant:
translation of the 1913 Michaelis--Menten paper,
{\it Biochem.} \textbf{50} 8264--8269, 2011.

\bibitem{LB34}
H.~Lineweaver and D.~Burk,
The Determination of Enzyme Dissociation Constants,
{\it J. Am. Chem. Soc.} \textbf{56} 658--666, 1934.

\bibitem{MM13}
L.~Michaelis and M.~L.~Menten,
Die Kinetik der Invertinwirkung,
{\it Biochem. Z.} \textbf{49} 333--369, 1913.

\bibitem{SS89}
L.~A.~Segel and M.~Slemrod,
The Quasi-Steady-State Assumption:
A Case Study in Perturbation,
{\it SIAM Review} \textbf{31(3)} 446--477, 1989.

\bibitem{TMS10}
M.~K.~Transtrum, B.~B.~Machta and J.~P.~Sethna,
Why are nonlinear fits to data so challenging?,
{\it ?Phys. Rev. Lett.} \textbf{104} 060201, 2010.

\bibitem{TMS11}
M.~K.~Transtrum, B.~B.~Machta and J.~P.~Sethna,
Geometry of nonlinear least squares
with applications to sloppy models and optimization,
{\it Phys. Rev. E} \textbf{83} 036701, 2011.

\bibitem{TQ14}
M.~K.~Transtrum and P.~Qiu,
Model reduction by manifold boundaries,
{\it ?Phys. Rev. Lett.} \textbf{113} 098701, 2014.

\bibitem{WCGBMBES06}
J.~J.~Waterfall, F.~P.~Casey, R.~N.~Gutenkunst, K.~S.~Brown, C.~R.~Myers,
P.~W.~Brouwer, V.~Elser and J.~P.~Sethna,
Sloppy-model universality class and the Vandermonde matrix,
{\it ?Phys. Rev. Lett.} \textbf{97} 150601, 2006.

\end{thebibliography}

\end{document}