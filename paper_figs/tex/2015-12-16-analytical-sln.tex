\documentclass[11pt]{article}

\usepackage{graphicx, subcaption, amsfonts, amsmath, amsthm, empheq,
  setspace, lscape, xcolor, physics}
\usepackage[top=1in, bottom=1in, left=1in, right=1in]{geometry}

% define some commands
% command to box formula
\newcommand*\widefbox[1]{\fbox{\hspace{2em}#1\hspace{2em}}}
\newlength\dlf
\newcommand\alignedbox[2]{
  % Argument #1 = before & if there were no box (lhs)
  % Argument #2 = after & if there were no box (rhs)
  &  % Alignment sign of the line
  {
    \settowidth\dlf{$\displaystyle #1$}  
    % The width of \dlf is the width of the lhs, with a displaystyle font
    \addtolength\dlf{\fboxsep+\fboxrule}  
    % Add to it the distance to the box, and the width of the line of the box
    \hspace{-\dlf}  
    % Move everything dlf units to the left, so that & #1 #2 is aligned under #1 & #2
    \boxed{#1 #2}
    % Put a box around lhs and rhs
  }
}

\newcommand\ER{Erd\H{o}s-R'{e}nyi}
\newcommand{\Forall}{\; \forall \;}
\DeclareMathOperator*{\argmin}{\arg\!\min}

% to ignore figures:
\renewcommand{\includegraphics}[2][]{\fbox{}}

% ceiling and floor symbols (use with asterisk: \ceil*{} or \floor*{}
\DeclarePairedDelimiter\ceil{\lceil}{\rceil}
\DeclarePairedDelimiter\floor{\lfloor}{\rfloor}

% change captions
\captionsetup{width=0.8\textwidth}
\captionsetup{labelformat=empty,labelsep=none}

% set paragraph indent length
\setlength\parindent{0pt}

% set folder for imported graphics
\graphicspath{ {../figs/dmaps/} }

\title{Analysis of anisotropic diffusion}

\begin{document}
\maketitle

\section{Analytical results}

To better understand how anisotropic diffusion affects the resulting
operator's eigenfunctions, we examine the PDE

\begin{align}
  &\frac{1}{D} \pdv{f}{t} = \pdv[2]{f}{x} + \epsilon \pdv[2]{f}{y} \\
\end{align}

with homogenous Neumann boundary conditions on $x,y \in
[0,a]\cross[0,b]$ and some arbitrary initial condition $f_0$

\begin{align}
  &f(0,x,y)=f_0 \\
  &\pdv{f}{x} (t,0,y)=\pdv{f}{x} (t,a,y)=\pdv{f}{y} (t,x,0)=\pdv{f}{x}
    (t,x,b)=0
\end{align}

Searching for a solution of the form $f=X(x)Y(y)T(t)$ via separation
of variables, we find three eigenvalue equations

\begin{alignat}{2}
  X'' &= \alpha X  & X'(0) = X'(a) = 0 \\
  Y'' &= \frac{\beta}{\epsilon} Y & Y'(0) = Y'(b) = 0 \\
  T' &= D (k+\lambda) T &
\end{alignat}

leading us to the general solution

\begin{align}
  f(x,y,t) = \sum_{i=0}^{\infty}\sum_{j=0}^{\infty} a_{ij} \cos(\frac{i \pi x}{a})
  \cos(\frac{j \pi y}{b}) e^{\lambda_{ij} t}
\end{align}

with 

\begin{align}
  \lambda_{ij} = -D \bigg( (\frac{i \pi}{a})^2 + \epsilon (\frac{j
  \pi}{b})^2 \bigg)
\end{align}

Thus, if $a = b = 1$, we expect the first $\sqrt{\frac{1}{\epsilon}}$
eigenfunctions to parameterize the direction of slower diffusion (in
this case $y$), followed by the fast direction, $x$. Subsequent
eigenfunctions should include a combination of both.

\section{Numerical results}

To confirm these results numerically, we sample the plane
$[0,1] \cross [0,1]$ to get a set of points $N$ $d_i = (x_i, y_i)$ and
then perform DMAPS on this planar dataset with the kernel

\begin{align}
  k(d_i,d_j) = e^{- \frac{(x_i - x_j)^2}{\epsilon} -
  \frac{(y_i - y_j)^2}{\epsilon^2}}
\end{align}

This should indeed correspond to a separation of diffusion timescales
between the $x$ and $y$ directions of order $\epsilon$. \\

Setting $\epsilon = 10^{-1}$, theory suggests the first
$\floor*{\sqrt{\frac{1}{\epsilon}}}=3$ eigenvectors should
parameterize the $y$ axis, eigenvector $4$ should parameterize $x$,
and the remaining will include some mixture of the two. This is indeed
what we find:

\begin{figure}[htbp]
  \centering
  \includegraphics[width=\linewidth]{./dmaps-plane-1}
  \caption{Dataset colored by DMAPS eigenvector $\Phi_1$,
    parameterizing $y$ axis}
\end{figure}

\begin{figure}[htbp]
  \centering
  \includegraphics[width=\linewidth]{./dmaps-plane-2}
  \caption{Dataset colored by DMAPS eigenvector $\Phi_2$,
    parameterizing $y$ axis}
\end{figure}

\begin{figure}[htbp]
  \centering
  \includegraphics[width=\linewidth]{./dmaps-plane-3}
  \caption{Dataset colored by DMAPS eigenvector $\Phi_3$,
    parameterizing $y$ axis}
\end{figure}

\begin{figure}[htbp]
  \centering
  \includegraphics[width=\linewidth]{./dmaps-plane-4}
  \caption{Dataset colored by DMAPS eigenvector $\Phi_4$,
    parameterizing $x$ axis}
\end{figure}

\begin{figure}[htbp]
  \centering
  \includegraphics[width=\linewidth]{./dmaps-plane-5}
  \caption{Dataset colored by DMAPS eigenvector $\Phi_5$,
    a combination of both directions}
\end{figure}

\clearpage

\section{Numerical results in higher dimensions}

It is unclear to me how to analytically approach a problem closer to
what we encounter when dealing with actual models, that is when

\begin{align}
  k(x_i,x_j) = \exp\bigg(- \frac{(x_i - x_j)^2}{\epsilon} - \alpha
  \frac{(f(x_i) - f(x_j))^2}{\epsilon^2}\bigg)
\end{align}

and the second term is dependent on the first. To start in this
pppdirection, we look at a dataset generated by
$d_i = (x_i, y_i, f(x_i, y_i))$ where $f(x_i, y_i) = x_i + y_i$. Thus
the dataset is a two dimensional plane embedded in three
dimensions. We set $\epsilon = 10^{-1}$ as before, but now include
$\alpha = 10^{-8}$ as well. We might expect the resulting eigenvectors
to be slightly perturbed versions of the $\alpha = 0$ case; and, if we
consider a rotation of the eigenvectors a small perturbation, one
could argue that this is indeed the outcome. I present below the dataset
$f(x_i, y_i) = z_i = x_i + y_i$ colored by the first several
eigenvectors of the DMAP. The first two appear to be rotations of the
traditional $\sin$ and $\cos$ functions, while later eigenvectors do
align exactly with the $x$ and $y$ axes.

\begin{figure}[htbp]
  \centering
  \includegraphics[width=\linewidth]{{{./dmaps-1d-dataspace-alpha-1e-08-eps-0.1-1}}}
  \caption{Dataset colored by DMAP eigenvector $\Phi_1$}
\end{figure}

\begin{figure}[htbp]
  \centering
  \includegraphics[width=\linewidth]{{{./dmaps-1d-dataspace-alpha-1e-08-eps-0.1-2}}}
  \caption{Dataset colored by DMAP eigenvector $\Phi_2$}
\end{figure}

\begin{figure}[htbp]
  \centering
  \includegraphics[width=\linewidth]{{{./dmaps-1d-dataspace-alpha-1e-08-eps-0.1-15}}}
  \caption{Dataset colored by DMAP eigenvector $\Phi_{15}$}
\end{figure}

\begin{figure}[htbp]
  \centering
  \includegraphics[width=\linewidth]{{{./dmaps-1d-dataspace-alpha-1e-08-eps-0.1-16}}}
  \caption{Dataset colored by DMAP eigenvector $\Phi_{16}$}
\end{figure}

\end{document}
