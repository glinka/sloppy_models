\documentclass[11pt]{article}

\usepackage{graphicx, subcaption, amsfonts, amsmath, amsthm, empheq, setspace, lscape, xcolor}
\usepackage[top=1in, bottom=1in, left=1in, right=1in]{geometry}

% define some commands
% command to box formula
\newcommand*\widefbox[1]{\fbox{\hspace{2em}#1\hspace{2em}}}
\newlength\dlf
\newcommand\alignedbox[2]{
  % Argument #1 = before & if there were no box (lhs)
  % Argument #2 = after & if there were no box (rhs)
  &  % Alignment sign of the line
  {
    \settowidth\dlf{$\displaystyle #1$}  
    % The width of \dlf is the width of the lhs, with a displaystyle font
    \addtolength\dlf{\fboxsep+\fboxrule}  
    % Add to it the distance to the box, and the width of the line of the box
    \hspace{-\dlf}  
    % Move everything dlf units to the left, so that & #1 #2 is aligned under #1 & #2
    \boxed{#1 #2}
    % Put a box around lhs and rhs
  }
}

\newcommand\ER{Erd\H{o}s-R'{e}nyi}
\newcommand{\Forall}{\; \forall \;}
\DeclareMathOperator*{\argmin}{\arg\!\min}

% change captions
\captionsetup{width=0.8\textwidth}
\captionsetup{labelformat=empty,labelsep=none}

% set paragraph indent length
\setlength\parindent{0pt}

% set folder for imported graphics
\graphicspath{ {../figs/} }

\title{Paper figures v2}

\begin{document}
\maketitle

\section{Dataspace example}

These are essentially clones of Antonios' plots made with Python

\begin{figure}[htbp]
  \centering
  \includegraphics[width=\linewidth]{./model-manifold/model-manifold-y0-coloring}
  \caption{Coloring by $\epsilon$ and $y(t_0)$}
\end{figure}

% \begin{figure}[htbp]
%   \centering
%   \includegraphics[width=\linewidth]{./model-manifold/model-manifold-eps-coloring}
%   \caption{Coloring by $\epsilon$ and $y(t_0)$}
% \end{figure}

\begin{figure}[htbp]
  \centering
  \includegraphics[width=\linewidth]{./model-manifold/model-manifold-inverse-eps-coloring}
  \caption{Coloring by $1/\epsilon$ and $y(t_0)$}
\end{figure}

\begin{figure}[htbp]
  \centering
  \includegraphics[width=\linewidth]{./model-manifold/model-manifold-ball-intersection}
  \caption{Intersection of ball with manifold}
\end{figure}

\begin{figure}[htbp]
  \centering
  \includegraphics[width=\linewidth]{./model-manifold/levelsets-nose}
  \caption{Level sets of the objective function in parameter space. As
  the objective function value increases, the contours become unbounded.}
\end{figure}

\clearpage

\section{DMAPS}

\subsection{2d vanilla}

Results from the reaction network model

\begin{figure}[htbp]
  \centering
  \includegraphics[width=\linewidth]{./rawlings/2d-dmaps}
  \caption{Plain dataset (left), colored by $\Phi_1$ (upper right), colored by $\Phi_2$ (lower right)}
\end{figure}

\clearpage

\subsection{3d with dataspace kernel}

Results from the custom kernel

\begin{figure}[htbp]
  \centering
  \includegraphics[width=\linewidth]{./rawlings/3d-dmaps}
  \caption{Plain dataset (left), colored by $k_{eff}$ (upper right), colored by $\Phi_1$ (lower right)}
\end{figure}

\subsection{Two effective, one neutral with dataspace kernel}

Here we use Antonios' model, fixing $b$ and varying $a$, $\epsilon$
and $\lambda$. By using a scaled dataspace kernel DMAPS we uncover
both effective parameters in $\Phi_1$ and $\Phi_2$ as shown below.

\begin{figure}[htbp]
  \centering
  \includegraphics[width=\linewidth]{./two-effective-one-neutral/dmaps-phi1}
  \caption{Coloring sloppy parameter combinations by $\Phi_1$}
\end{figure}

\begin{figure}[htbp]
  \centering
  \includegraphics[width=\linewidth]{./two-effective-one-neutral/dmaps-phi2}
  \caption{Coloring sloppy parameter combinations by $\Phi_2$}
\end{figure}


% \bibliographystyle{plain}
% \bibliography{literature.bib}

\end{document}
