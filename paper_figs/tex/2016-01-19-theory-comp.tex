\documentclass[11pt]{article}

\usepackage{graphicx, subcaption, amsfonts, amsmath, amsthm, empheq, setspace, lscape, xcolor}
\usepackage[top=1in, bottom=1in, left=1in, right=1in]{geometry}

% define some commands
% command to box formula
\newcommand*\widefbox[1]{\fbox{\hspace{2em}#1\hspace{2em}}}
\newlength\dlf
\newcommand\alignedbox[2]{
  % Argument #1 = before & if there were no box (lhs)
  % Argument #2 = after & if there were no box (rhs)
  &  % Alignment sign of the line
  {
    \settowidth\dlf{$\displaystyle #1$}  
    % The width of \dlf is the width of the lhs, with a displaystyle font
    \addtolength\dlf{\fboxsep+\fboxrule}  
    % Add to it the distance to the box, and the width of the line of the box
    \hspace{-\dlf}  
    % Move everything dlf units to the left, so that & #1 #2 is aligned under #1 & #2
    \boxed{#1 #2}
    % Put a box around lhs and rhs
  }
}

\newcommand\ER{Erd\H{o}s-R'{e}nyi}
\newcommand{\Forall}{\; \forall \;}
\DeclareMathOperator*{\argmin}{\arg\!\min}

% change captions
\captionsetup{width=0.8\textwidth}
\captionsetup{labelformat=empty,labelsep=none}

% set paragraph indent length
\setlength\parindent{0pt}

% set folder for imported graphics
\graphicspath{ {../figs/} }

\title{A comparison of DMAPS and the Neumann heat kernel}

\begin{document}
\maketitle

Theoretically the DMAPS eigenvalues and eigenvectors converge to the
eigenvalues and eigenfunctions of certain differential operators,
namely those of the Laplace-Beltrami operator on the manifold, with
Neumann boundary conditions. Below we compare the eigenvalues and
eigenvectors computed with DMAPS to the corresponding analytical
eigenvalues and eigenfunctions.

\section*{Eigenvalues}

Theory shows that

\begin{align*}
  \lambda_i^{\frac{t}{\epsilon}} \rightarrow e^{-t \hat{\nu_i}^2}
\end{align*}

where $\lambda$ corresponds to DMAPS eigenvalues and $\hat{\nu}$ to
eigenvalues of the Laplace operator. To be precise, $\hat{\nu}$ is actually
an eigenvalue of a scaled version of the operator $\Delta_0 = \alpha
\Delta$ where $\Delta$ is the true Laplace-Beltrami operator. Lafon
gives a vague expression for $\alpha$ that I couldn't work out
computationally, so I simply estimated it as $\alpha = 2$. This
value holds for many different domains and densities of points which
suggests it's close to the true value. Thus if $\nu$ represents the
eigenvalues of the unscaled L-B operator, we expect

\begin{align*}
  \lambda_i^{\frac{t}{\epsilon}} \rightarrow e^{-\alpha^2 t \nu_i^2}
\end{align*}

Sampling points on a rectangular domain $x \in [0, a]$, $y \in [0,b]$
and comparing DMAPS computed eigenvalues with the theoretical values
yields good agreement as shown below. We set $t = \epsilon$.

\begin{figure}[htbp]
  \centering
  \includegraphics[width=\linewidth]{./analytical-dmaps-comparison/eigvals-comparison}
  \caption{Comparison of DMAPS eigenvalues (red line) with the
    analytical value of $e^{-\alpha \epsilon \nu_i^2}$ (blue dots)}
\end{figure}

\clearpage

\section{Eigenvectors}

Theory also suggets that the DMAP eigenvectors converge to the L-B
eigenfunctions. To compare these quantities, we sampled the
theoretical eigenfunctions over the domain and calculated the
difference between these values and the correpsonding DMAPS
eigenvector arising from the data. Specifically, if
$\Phi_i \in \mathbb{R}^n$ is the $i^{th}$ DMAPS eigenvector with a
norm equal to one, and
$F_i \in \mathbb{R}^n$ contains the values of the $i^{th}$
eigenfunction evaluated at each of the $n$ points in the DMAPS
dataset, we look at the relative error defined by

\begin{align*}
  e_i = \frac{\|\|F_i\|\Phi_i - F_i\|}{\|F_i\|}
\end{align*}

We scale $\Phi_i$ by the norm of $F_i$ to account for the arbitrary
scaling of the eigenvector. Examining the resulting errors shows good
agreement between computation and theory.

\begin{figure}[htbp]
  \centering
  \includegraphics[width=\linewidth]{./analytical-dmaps-comparison/eigvects-error}
  \caption{Relative error in the DMAPS estimate of the theoretical eigenfunction}
\end{figure}

The following simply show the dataset colored by these eigenvectors
and eigenfunctions allowing for quick qualitative comparison.

\clearpage

1st eigenvector/eigenfunction

\begin{figure}[htbp]
  \centering
  \includegraphics[width=0.65\linewidth]{./analytical-dmaps-comparison/eigvect1}
\end{figure}

\begin{figure}[htbp]
  \centering
  \includegraphics[width=0.65\linewidth]{./analytical-dmaps-comparison/eigfn1}
\end{figure}

\clearpage

2nd eigenvector/eigenfunction

\begin{figure}[htbp]
  \centering
  \includegraphics[width=0.65\linewidth]{./analytical-dmaps-comparison/eigvect2}
\end{figure}

\begin{figure}[htbp]
  \centering
  \includegraphics[width=0.65\linewidth]{./analytical-dmaps-comparison/eigfn2}
\end{figure}

\clearpage

3rd eigenvector/eigenfunction

\begin{figure}[htbp]
  \centering
  \includegraphics[width=0.65\linewidth]{./analytical-dmaps-comparison/eigvect3}
\end{figure}

\begin{figure}[htbp]
  \centering
  \includegraphics[width=0.65\linewidth]{./analytical-dmaps-comparison/eigfn3}
\end{figure}

\clearpage

4th eigenvector/eigenfunction

\begin{figure}[htbp]
  \centering
  \includegraphics[width=0.65\linewidth]{./analytical-dmaps-comparison/eigvect4}
\end{figure}

\begin{figure}[htbp]
  \centering
  \includegraphics[width=0.65\linewidth]{./analytical-dmaps-comparison/eigfn4}
\end{figure}

\clearpage

5th eigenvector/eigenfunction

\begin{figure}[htbp]
  \centering
  \includegraphics[width=0.65\linewidth]{./analytical-dmaps-comparison/eigvect5}
\end{figure}

\begin{figure}[htbp]
  \centering
  \includegraphics[width=0.65\linewidth]{./analytical-dmaps-comparison/eigfn5}
\end{figure}

\clearpage

6th eigenvector/eigenfunction

\begin{figure}[htbp]
  \centering
  \includegraphics[width=0.65\linewidth]{./analytical-dmaps-comparison/eigvect6}
\end{figure}

\begin{figure}[htbp]
  \centering
  \includegraphics[width=0.65\linewidth]{./analytical-dmaps-comparison/eigfn6}
\end{figure}

\clearpage

7th eigenvector/eigenfunction

\begin{figure}[htbp]
  \centering
  \includegraphics[width=0.65\linewidth]{./analytical-dmaps-comparison/eigvect7}
\end{figure}

\begin{figure}[htbp]
  \centering
  \includegraphics[width=0.65\linewidth]{./analytical-dmaps-comparison/eigfn7}
\end{figure}

\clearpage

8th eigenvector/eigenfunction

\begin{figure}[htbp]
  \centering
  \includegraphics[width=0.65\linewidth]{./analytical-dmaps-comparison/eigvect8}
\end{figure}

\begin{figure}[htbp]
  \centering
  \includegraphics[width=0.65\linewidth]{./analytical-dmaps-comparison/eigfn8}
\end{figure}

% \bibliographystyle{plain}
% \bibliography{literature.bib}

\end{document}