\documentclass[11pt]{article}

\usepackage{graphicx, subcaption, amsfonts, amsmath, amsthm, empheq, setspace, lscape, xcolor}
\usepackage[top=1in, bottom=1in, left=1in, right=1in]{geometry}

% define some commands
% command to box formula
\newcommand*\widefbox[1]{\fbox{\hspace{2em}#1\hspace{2em}}}
\newlength\dlf
\newcommand\alignedbox[2]{
  % Argument #1 = before & if there were no box (lhs)
  % Argument #2 = after & if there were no box (rhs)
  &  % Alignment sign of the line
  {
    \settowidth\dlf{$\displaystyle #1$}  
    % The width of \dlf is the width of the lhs, with a displaystyle font
    \addtolength\dlf{\fboxsep+\fboxrule}  
    % Add to it the distance to the box, and the width of the line of the box
    \hspace{-\dlf}  
    % Move everything dlf units to the left, so that & #1 #2 is aligned under #1 & #2
    \boxed{#1 #2}
    % Put a box around lhs and rhs
  }
}

\newcommand\ER{Erd\H{o}s-R'{e}nyi}
\newcommand{\Forall}{\; \forall \;}
\DeclareMathOperator*{\argmin}{\arg\!\min}

% change captions
\captionsetup{width=0.8\textwidth}
\captionsetup{labelformat=empty,labelsep=none}

% set paragraph indent length
\setlength\parindent{0pt}

% set folder for imported graphics
\graphicspath{ {../figs/dmaps/} }

\title{Diffusion on plane in embedded in $\mathbb{R}^4$}
\author{}

\begin{document}
\maketitle

All of the following were run with $\epsilon = 10^{-1}$ and $\alpha =
10$ so we have essentially used the kernel

\begin{align*}
  k(d_i, d_j) = \exp(- \frac{\| d_i - d_j\|^2}{\epsilon})
\end{align*}


for $d_i = (x_i, y_i, \lambda_1 x_i, \lambda_2 y_i)$. $\lambda_1$ was
always set to $\sqrt{\frac{1}{2}}$.

\section{$\lambda_2 = \sqrt{\frac{1}{2}}$}

\begin{figure}[htbp]
  \centering
  \includegraphics[width=\linewidth]{{{./dmaps-2d-dataspace-lam2-0.707106781187-alpha-10-eps-0.1-1}}}
  \caption{Coloring dataset (image and preimage) with $\Phi_1$}
\end{figure}

\begin{figure}[htbp]
  \centering
  \includegraphics[width=\linewidth]{{{./dmaps-2d-dataspace-lam2-0.707106781187-alpha-10-eps-0.1-2}}}
  \caption{Coloring dataset (image and preimage) with $\Phi_2$}
\end{figure}

\clearpage

\section{$\lambda_2 = \sqrt{\frac{2}{3}}$}

\begin{figure}[htbp]
  \centering
  \includegraphics[width=\linewidth]{{{./dmaps-2d-dataspace-lam2-0.816496580928-alpha-10-eps-0.1-1}}}
  \caption{Coloring dataset (image and preimage) with $\Phi_1$}
\end{figure}

\begin{figure}[htbp]
  \centering
  \includegraphics[width=\linewidth]{{{./dmaps-2d-dataspace-lam2-0.816496580928-alpha-10-eps-0.1-2}}}
  \caption{Coloring dataset (image and preimage) with $\Phi_2$}
\end{figure}


% \bibliographystyle{plain}
% \bibliography{literature.bib}

\end{document}
