\documentclass[11pt]{article}

\usepackage{graphicx, subcaption, amsfonts, amsmath, amsthm, empheq, setspace, lscape, xcolor}
\usepackage[top=1in, bottom=1in, left=1in, right=1in]{geometry}

% define some commands
% command to box formula
\newcommand*\widefbox[1]{\fbox{\hspace{2em}#1\hspace{2em}}}
\newlength\dlf
\newcommand\alignedbox[2]{
  % Argument #1 = before & if there were no box (lhs)
  % Argument #2 = after & if there were no box (rhs)
  &  % Alignment sign of the line
  {
    \settowidth\dlf{$\displaystyle #1$}  
    % The width of \dlf is the width of the lhs, with a displaystyle font
    \addtolength\dlf{\fboxsep+\fboxrule}  
    % Add to it the distance to the box, and the width of the line of the box
    \hspace{-\dlf}  
    % Move everything dlf units to the left, so that & #1 #2 is aligned under #1 & #2
    \boxed{#1 #2}
    % Put a box around lhs and rhs
  }
}

\newcommand\ER{Erd\H{o}s-R'{e}nyi}
\newcommand{\Forall}{\; \forall \;}
\DeclareMathOperator*{\argmin}{\arg\!\min}

% change captions
\captionsetup{width=0.8\textwidth}
\captionsetup{labelformat=empty,labelsep=none}

% set paragraph indent length
\setlength\parindent{0pt}

% set folder for imported graphics
\graphicspath{ {../figs/} }

\title{Nonlinear contour of simple reaction scheme}

\begin{document}
\maketitle

To clarify, the reaction we study is given by

\begin{align*}
  A \xrightleftharpoons[k_{-1}]{k_1} B, \; B \xrightarrow[]{k_2} C
\end{align*}

and we focus on the parameter regime $k_2 \gg k_1$ and $k_{-1} \gg
k_1$ corresponding to a small value of $\epsilon = \frac{k_1}{k_{-1} +
  k_2}$ and an effective parameter of

\begin{align*}
  k_{eff} = \frac{k_1 k_2}{k_{-1} + k_2}
\end{align*}

the validity of which was analytically verified in Antonios' most
recent notes. \\

There were two points to confirm numerically: 

\begin{enumerate}
\item Is the neutral set nonlinear?
\item Does the vector normal to the neutral set point in a direction
  of varying $k_{eff}$?
\end{enumerate}

To test the first, we sample parameter space in a region in which
$\epsilon \ll 1$, specifically:

\begin{align*}
  k_1 &\in [10^{-4}, 10^0] \\
  k_{-1} &\in [10^1, 10^4] \\
  k_2 &\in [10^1, 10^4] \\
\end{align*}

where the true parameter value was set to

\begin{align*}
  k_1 &= 1 \\
  k_{-1} &= 10^3 \\
  k_2 &= 10^3 \\
\end{align*}

We then keep only those parameter combinations at which the
least-squares objective function value is less than $10^{-3}$, and are
left with the following plots. Figs. (\ref{fig:r1}) to
(\ref{fig:r2-of}) all plot the same data at different viewing
angles. Figs. (\ref{fig:r1}) to (\ref{fig:r3}) are colored by
$k_{eff}$ value, while Fig. (\ref{fig:r2-of}) is colored by objective
function value. The main point in these figures is that
the level set is indeed nonlinear. \\

While colorings alone strongly suggest that $k_{eff}$ changes along the
direction normal to the surface, we can confirm this by calculating an
actual normal vector to the surface (around the point circled in the
figures) using PCA, and comparing this vector with the gradient of the
$k_{eff}$ (or equivalently, up to scaling, the gradient of the
objective function). At the point specified in the plot $p$, we calculate
a normal vector of

\begin{align*}
  v(p) = [0.88, -0.32, 0.33]
\end{align*}

and a gradient of 

\begin{align*}
  \nabla k_{eff} (p) = [0.87, -0.34, 0.34]
\end{align*}

for an error of $\| v - \nabla k_{eff} \| = 0.023$. \\

Thus not only is the contour nonlinear, but the direction normal to
the surface does indeed point in the direction of changing $k_{eff}$
as desired.

\begin{figure}[htbp]
  \centering
  \includegraphics[width=\linewidth]{./r1}
  \caption{Neutral set colored by $k_{eff}$ \label{fig:r1}}
\end{figure}

\begin{figure}[htbp]
  \centering
  \includegraphics[width=\linewidth]{./r2}
  \caption{Neutral set colored by $k_{eff}$ \label{fig:r2}}
\end{figure}

\begin{figure}[htbp]
  \centering
  \includegraphics[width=\linewidth]{./r3}
  \caption{Neutral set colored by $k_{eff}$ \label{fig:r3}}
\end{figure}

\begin{figure}[htbp]
  \centering
  \includegraphics[width=\linewidth]{./r2-of}
  \caption{Neutral set colored by objective function value \label{fig:r2-of}}
\end{figure}

% \bibliographystyle{plain}
% \bibliography{literature.bib}

\end{document}
