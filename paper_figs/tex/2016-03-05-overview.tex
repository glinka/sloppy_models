\documentclass[11pt]{article}

\usepackage{graphicx, subcaption, amsfonts, amsmath, amsthm, empheq,
  setspace, lscape}
\usepackage[top=1in, bottom=1in, left=1in, right=1in]{geometry}

% define some commands
% command to box formula
\newcommand*\widefbox[1]{\fbox{\hspace{2em}#1\hspace{2em}}}
\newlength\dlf
\newcommand\alignedbox[2]{
  % Argument #1 = before & if there were no box (lhs)
  % Argument #2 = after & if there were no box (rhs)
  &  % Alignment sign of the line
  {
    \settowidth\dlf{$\displaystyle #1$}  
    % The width of \dlf is the width of the lhs, with a displaystyle font
    \addtolength\dlf{\fboxsep+\fboxrule}  
    % Add to it the distance to the box, and the width of the line of the box
    \hspace{-\dlf}  
    % Move everything dlf units to the left, so that & #1 #2 is aligned under #1 & #2
    \boxed{#1 #2}
    % Put a box around lhs and rhs
  }
}

\newcommand{\ps}{\mathrm{\Theta}}
\newcommand{\p}{\theta}
\newcommand{\eps}{\varepsilon}
\newcommand{\be}{\begin{equation}}
\newcommand{\ee}{\end{equation}}
\newcommand\ER{Erd\H{o}s-R'{e}nyi}
\newcommand{\Forall}{\; \forall \;}
\DeclareMathOperator*{\argmin}{\arg\!\min}

% change captions
\captionsetup{width=0.8\textwidth}
% \captionsetup{labelformat=empty,labelsep=none} 

% set paragraph indent length
\setlength\parindent{0pt}

% set folder for imported graphics
\graphicspath{ {../figs/} }

\title{Overview of sloppy models}

\begin{document}
\maketitle

Below is a brief overview of the various models we could employ to display different varieties of sloppiness.

\section{Singularly perturbed}

We have two singularly perturbed systems: the first is an ODE
constructed to have a nonlinear dataspace, the second arises from a
basic, reversible reaction mechanism involving three components: A, B
and C.

\subsection{Singularly perturbed ODE, sloppy initial conditions}

The system takes the form

\be
\begin{array}{rcl}
 \dot{x} &=& y - \lambda x ,
\vspace*{1mm}\\
 \eps \dot{y} &=& \eps x - \displaystyle\left(1+\frac{10}{1.5-\sin y}\right) y ,
\end{array}
\ \mbox{supplemented with} \
\begin{array}{rcl}
 x(0) &=& x_0 ,
\vspace*{1mm}\\
 y(0) &=& y_0 .
\end{array}
\label{elem-ODE}
\ee

This model has a unique, globally attracting steady state at the
origin whose stability specifics are controlled by $\eps$ and
$\lambda$.  All four of $(\eps,\lambda,x_0,y_0)$ can be viewed as
parameters, but we reduce their number to $M=2$ by setting
$\p = (\eps,y_0)$ and fixing $\lambda= 2$ and $x_0 = 1$.  For our
observed model response, we choose

\be
 \mu(\p) = \big( y(t_1;\p) \,,\, y(t_2;\p) \,,\, y(t_3;\p) \big) ,
\ \mbox{for fixed time instants} \
 0 < t_1 < t_2 < t_3 .
\label{elem-mu}
\ee

It follows that the $2-$D observed model manifold is embedded in $3-$D
Euclidean space ($N=3$).  A segment of this highly nonlinear manifold
is plotted in Fig.~\ref{f.elem.ex.1}. \\

Depending on our parameter values, we may encounter parameter
sloppiness in both $\eps$ and $y_0$. This reveals itself by examining
level sets of the objective function. Close enough to the minimum, we
necessarily have elliptical contours in parameter space. However, in a
sloppy regime, these ellipses quickly become unbounded regions that
appear as noses. Essentially, as $\eps$ decreases, a larger and larger
span of $y_0$ values fit the data within some given
tolerance. Fig.~\ref{f.noses} shows this phenomenon.

\begin{figure}[ht!]
  \begin{subfigure}[t]{0.49\textwidth}
    \centering
    \includegraphics[width=\textwidth]{./model-manifold/model-manifold-y0-coloring}
    \subcaption{Colored by $y_0$}
  \end{subfigure}
  \begin{subfigure}[t]{0.49\textwidth}
    \centering
    \includegraphics[width=\textwidth]{./model-manifold/model-manifold-eps-coloring}
    \subcaption{Colored by $\epsilon$}
  \end{subfigure} %
  \caption{The observed model manifold for
    system~\eqref{elem-ODE} in $3-$D \emph{observed model space}. The
    two parameters here are $\epsilon$ and $y_0$, and the map $\mu$
    from parameter to (observed) model space is given
    in~\eqref{elem-mu}. The manifold has been colored by each
    parameter to visualize how these vary on it. The manifold bottom
    corresponds to the singularly perturbed regime $0 < \eps \ll 1$,
    see left panel. In that regime, widely different initial
    conditions yield nearly identical model responses, see right
    panel. \label{f.elem.ex.1}}
\end{figure}

\begin{figure}[htbp]
  \centering
  \includegraphics[width=\linewidth]{./rawlings/keff-with-levelsets}
  \caption{Different level sets of the objective function in a sloppy
    regime. As $1/\eps$ increases, a larger range of $y_0$ values give
    acceptable fits.}
\end{figure}

\subsection{ABC Reaction}

The mechanism is given by

\begin{align*}
  A \xrightleftharpoons[k_{-1}]{k_1} B, \; B \xrightarrow[]{k_2} C
\end{align*}

under the assumption that species $B$ is in quasi equilibrium. This requires

\begin{align*}
  \frac{dC_B}{dt} &= k_1 C_A - (k_{-1} + k_2) C_B = 0 \\
  &\rightarrow C_B = C_A \frac{k_1}{k_{-1} + k_2}
\end{align*}

leading to the analytical, QSSA solution for $C_C$

\begin{align*}
  C_C = C_{A_0}(1 - e^{-\frac{k_1 k_2}{k_{-1} + k_2} t})
\end{align*}

Here we see the effective parameter

\begin{align*}
  k_{eff} = \frac{k_1 k_2}{k_{-1} + k_2}
\end{align*}

will create nonlinear level sets in parameter space instead of simple planes.

as shown below, along with DMAPS parameterization of the sloppy parameter set.

\begin{figure}[htbp]
  \centering
  \includegraphics[width=\linewidth]{./rawlings/keff-neutralset}
  \caption{Dataset of sloppy parameter combinations overlayed with
    with a surface of $k_{eff} =0.502$. All points satisfy $c(\theta)
    \le 10^{-6}$. \label{f.keff-neutralset}}
\end{figure}

\begin{figure}[ht!]
  \begin{subfigure}[t]{0.49\textwidth}
    \centering
    \includegraphics[width=\linewidth]{./rawlings/keff-with-levelsets-r3}
  \end{subfigure}
  \begin{subfigure}[t]{0.49\textwidth}
    \centering
    \includegraphics[width=\linewidth]{./rawlings/keff-with-levelsets-r4}
  \end{subfigure} %
  \caption{Dataset of sloppy parameter combinations overlayed with
    with a surfaces of constant $k_{eff}$. All points satisfy $c(\theta)
    \le 10^{-3}$. Note that, while Fig. (\ref{f.keff-neutralset}) is
    a two-dimensional surface, at larger objective function values we
    begin to see three-dimensional features. \label{f.keff-levelsets}}
\end{figure}


\begin{figure}[htbp]
  \centering
  \caption{Dataset of sloppy parameter combinations overlayed with
    with a surface of $k_{eff} =0.502$. All points satisfy $c(\theta)
    \le 10^{-6}$.}
\end{figure}

As expected, it maps out a two-dimensional surface in parameter space over which $k_{eff}$ is nearly constant ($k_{eff} \in (0.4, 1.0)$ in the dataset above).

When DMAPS was applied, the first two eigenvectors parameterized the surface as hoped. This is shown in the two figures below.

% \begin{figure}[htbp]
%   \centering
%   \includegraphics[width=\linewidth]{abc-dmap1}
%   \caption{Coloring the dataset by the first DMAP parameter/coordinate value}
% \end{figure}

% \begin{figure}[htbp]
%   \centering
%   \includegraphics[width=\linewidth]{abc-dmap2}
%   \caption{Coloring the dataset by the second DMAP parameter/coordinate value}
% \end{figure}

Additionally, we can use the mixed DMAPS kernel to find the important
parameter $k_{eff}$ if we apply it to a dataset arising from a larger
ball in model space (which can also be thought of as increasing the
tolerance of our fitting routine). Fig. \ref{fig:dmaps-mixed} shows
that the third eigenvector maps to $k_{eff}$.

\begin{figure}[htbp]
  \centering
  \includegraphics[width=\linewidth]{./rawlings/3d-dmaps}
  \caption{Coloring the enlarged parameter set by $k_{eff}$ (top
    right) and by the third DMAPS eigenvector (bottom right). The
    colors are similar in both, suggesting the third eigenvector is
    one-to-one with $k_{eff}$. \label{fig:dmaps-mixed}}
\end{figure}

\subsection{The singular regime}

Ideally, we would operate where $k_2 \gg k_1, \; k_{-1}$. However,
limiting our analysis to this region leads to

\begin{align*}
  k_{eff} &= \frac{k_1}{\frac{k_{-1}}{k_2} + 1}
  & \approx k_1
\end{align*}

which removes the nonlinearity from the contours. If we could ease our
restrictions to just $k_2 \gg k_1$ we do recover the desired curvature.

\section{Singularly and regularly perturbed}

Here we turn to Antonios' model which includes all varieties of sloppiness we wish to show: singular and regular perturbation parameters and sloppy initial conditions. This system is ideal for a nonlinear transformation of parameters.

\subsection{Antonios' Model}

We start with the linear ODEs

\begin{align*}
  X &= -\lambda X \\
  \epsilon Y &= -Y
\end{align*}

and then transform it into nonlinear $(x, y)$ via 

\begin{align*}
  x &= X + \phi(y) \\
  y &= Y + \mu(x) \\
\end{align*}

We are free to choose $\phi(y)$ and $\mu(x)$, which control the shape of the fast and slow manifold, respectively. Thus, if we set $\mu = a x^2$ and $\phi = b y^2$ we find parabolic slow and fast manifolds, and additionally we've introduced a sloppy regular perturbation parameter $b$. Additionally, initial conditions lying along a given fast manifold will be sloppy. Sloppiness in $\epsilon$/$\lambda$, $a$/$b$ and $x_0$/$y_0$ are shown in the three figures below.

\begin{figure}[htbp]
  \centering
  \includegraphics[width=\linewidth]{./zagaris/sing-pert-init-cond-contours}
  \caption{$x_0$/$y_0$ plane colored by objective function value. The contours follow the fast manifold $x=y^2$.}
\end{figure}

\begin{figure}[htbp]
  \centering
  \includegraphics[width=\linewidth]{./zagaris/sing-pert-phase-plane}
  \caption{The phase plane showing parabolic fast and slow
    trajectories on their approach to the stable steady state at the origin.}
\end{figure}


\subsubsection{Nonlinear parameter transformation}

By holding the sloppy parameters $(\epsilon, b)$ constant we are left
with a system in which both remaining parameters, $\lambda$ and $a$
are not sloppy. If we apply two iterations of the H\'{e}non map to a
set of $(\lambda, a)$ values that fit some base model within a given
tolerance, we can make the transformed parameters $(\theta_1,
\theta_2)$ appear sloppy.

\begin{figure}[ht!]
  \begin{subfigure}[t]{0.49\textwidth}
    \centering
    \includegraphics[width=\textwidth]{./transformed-params/transformed-params-fromoptimization-insert}
    \subcaption{Parameter values found through repeated fitting of
      the model with parameters transformed via the H\`{e}non map. \label{f.henon}}
  \end{subfigure}
  \begin{subfigure}[t]{0.49\textwidth}
    \centering
    \includegraphics[width=\textwidth]{./transformed-params/inverted-params}
    \subcaption{Original parameter values found by inverting the
      collection in the left panel.  \label{f.henon-inverse}}
  \end{subfigure} %
  \caption{The set $\ps_\delta$ corresponding to the point
    $(\alpha_*,\lambda_*) = (1,1)$. The absence of
    sloppiness is evident in the right panel, in which $\ps_\delta$ is
    plotted in terms of the original parameter set
    $(\alpha,\lambda)$. To the contrary, the same domain plotted in
    terms of the transformed parameters $(\p_1,\p_2)$ appears bent and
    sloppy. \label{f.transf-params}}
\end{figure}

\subsubsection{Two effective parameters, one neutral}

We can also hold $b$ constant to obtain a model with two effective
parameters $\lambda$ and $a$, and one neutral, $\epsilon$. This gives
us another context in which we can demonstrate the advantage the mixed DMAPS
kernel offers over the standard gaussian. Here, we uncover both
effective parameters $\lambda$ and $a$ in the first two DMAPS
eigenvectors despite the much larger variation in $\eps$. This kernel
is given by

\begin{align*}
  k(\theta_i, \theta_j) = \exp(- \frac{\|\theta_i -
  \theta_j\|^2}{\eps}  - \frac{\|f(\theta_i) - f(\theta_j)\|^2}{\eps^2})
\end{align*}

The results of applying this variant of DMAPS to a set of $(\lambda,
a, \eps)$ parameters is shown below.


\begin{figure}[ht!]
  \begin{subfigure}[t]{0.49\textwidth}
    \centering
    \includegraphics[width=\textwidth]{./two-effective-one-neutral/dmaps-phi1}
    \subcaption{Coloring sloppy parameter combinations by $\Phi_1$}
  \end{subfigure}
  \begin{subfigure}[t]{0.49\textwidth}
    \centering
    \includegraphics[width=\textwidth]{./two-effective-one-neutral/dmaps-phi2}
    \subcaption{Coloring sloppy parameter combinations by $\Phi_2$}
  \end{subfigure} %
  \caption{The set $\ps_\delta$ corresponding to the point
    $(\alpha_*,\lambda_*) = (1,1)$. The absence of
    sloppiness is evident in the right panel, in which $\ps_\delta$ is
    plotted in terms of the original parameter set
    $(\alpha,\lambda)$. To the contrary, the same domain plotted in
    terms of the transformed parameters $(\p_1,\p_2)$ appears bent and
    sloppy. \label{f.transf-params}}
\end{figure}

% \bibliographystyle{plain}
% \bibliography{literature.bib}

\end{document}
