\documentclass[11pt]{article}
\usepackage{graphicx, subcaption, amsfonts, amsmath, amsthm, empheq, lscape}
\usepackage[top=0.8in, bottom=0.8in, left=1in, right=1in]{geometry}
\graphicspath{ {./figs/} }
\pagestyle{plain}
\begin{document}
\title{Objective function contours}
\author{}
\date{}
\maketitle

Given the system

\begin{align}
  \label{eq1}
  \frac{d}{dt} \begin{bmatrix} x_1 \\ x_2 \end{bmatrix} = \begin{bmatrix} -1 & 0 \\ 0 & -\sigma \end{bmatrix} \begin{bmatrix} x_1 \\ x_2 \end{bmatrix}
\end{align}

we generate ten points of noise-free data, shown in Fig. (\ref{fig1}) below.

\begin{figure}[h!]
  \centering
  \includegraphics[width=1.0\textwidth]{phase_plot_sigma1000}
  \caption{Phase plot of Eq. (\ref{eq1}) with $\sigma=1000$}
  \label{fig1}
\end{figure}

We fit this data to Eq. (\ref{eq1}) through a least-squares fit parameter fit with objective function

\begin{align}
  \label{eq2}
  f(x_2(0), \sigma) = \sum\limits_{i=1}^N \| \hat{x}(t_i; x_2(0), \sigma) - x(t_i) \|^2
\end{align}

in which $\hat{x}(t_i; x_2(0), \sigma)$ is the state of the system at time $t_i$ given parameters $x_2(0)$ and $\sigma$. Note that we consider $x_1(0)$ a fixed parameter.

We are then interested in contours of Eq. (\ref{eq2}) which we find by solving

\begin{align}
  \label{eq3}
  f(x_2(0), \sigma) - c = 0
\end{align}

using a Newton-Rhapson scheme with a Poincare cut. Once a value has been converged to, we use simple parameter continuation to enumerate further points on the contour $f = c$. These are given in Fig. (\ref{fig2}) below.

\begin{landscape}
\begin{figure}[h!]
  \centering
  \centering
  \includegraphics[width=1.5\textwidth]{contours_nofillb}
  \caption{Contours of $f$}
  \label{fig2}
\end{figure}

\begin{figure}[h!]
  \centering
  \includegraphics[width=1.5\textwidth]{contours_fill}
  \caption{Contours of $f$}
  \label{fig3}
\end{figure}
\end{landscape}

\end{document}
