\documentclass[11pt]{article}
\usepackage{graphicx, subcaption, amsfonts, amsmath, amsthm, empheq, lscape}
\usepackage[top=0.8in, bottom=0.8in, left=1in, right=1in]{geometry}

\newcommand{\R}{\mathrm{R}}
\newcommand{\N}{\mathrm{N}}
\newcommand{\T}{\mathrm{T}}
\newcommand{\X}{\mathrm{X}}
\newcommand{\D}{\mathrm{D}}
\newcommand{\V}{\mathcal{V}}
\newcommand{\M}{\mathcal{M}}
\newcommand{\ip}[2]{\langle#1,#2\rangle}
\newcommand{\abs}[1]{\vert#1\vert}
\newcommand{\nv}[1]{\vert#1\vert}
\newcommand{\nx}[1]{\vert\vert#1\vert\vert}
\newcommand{\dv}[2]{d(#1,#2)}
\newcommand{\eps}{\varepsilon}

\newcommand{\be}{\begin{equation}}
\newcommand{\ee}{\end{equation}}
\newcommand{\dst}{\displaystyle}

\graphicspath{ {../figs/} }
\pagestyle{plain}
\begin{document}
\title{Michaelis-Menten Sloppiness}
\author{Model description: Prof. Zagaris\\ Numerical results: Alexander Holiday}
\date{}
\maketitle

\section{The model}

Consider the chemical pathway
%
\[
 {\rm S + E}
\
 \xrightleftharpoons[k_{-1}]{k_1}
\
 {\rm C}
\
 \xrightarrow{k_2}
\
 {\rm P + E} .
\label{s2c2p}
\]
%
The initial concentrations are
$S_0$, $E_0$, $C_0$ and $P_0$,
and the system has two exact conservation laws,
%
\[
 S+C+P = S_0+C_0+P_0 = S_T
\quad\mbox{and}\quad
 C+E = C_0+E_0 = E_T .
\]
%
The ODEs describing the evolution of the constituents are
%
\be
\begin{array}{rclcl}
 S' &=& -k_1 E S + k_{-1} C ,
\\
 C' &=& \ \ \, k_1 E S - (k_{-1} + k_2) C ,
\\
 E' &=& -k_1 E S + (k_{-1} + k_2) C ,
\\
 P' &=& \ \ \, k_2 C .
\end{array}
\label{SCEP-ODE}
\ee
%
Two of these may be eliminated by virtue of the conservation laws.
These are traditionally the equations for $E$ and $P$,
so that
%
\be
\begin{array}{rclcl}
 S' &=& -k_1 (E_T-C) S + k_{-1} C ,
\\
 C' &=& \ \ \, k_1 (E_T-C) S - (k_{-1} + k_2) C .
\end{array}
\label{SC-ODE}
\ee
%
Typically, $C_0$ and $P_0$ equal zero,
so that $S_0$ and $E_0$ fully determine $S_T$ and $E_T$,
%
\be
 S_0 = S_T
\quad\mbox{and}\quad
 C_0 = 0 .
\label{SC-IP}
\ee
%
We will follow this convention here,
and thus consider a problem with five parameters:
the three kinetic constants $k_{-1}$, $k_1$ and $k_2$,
on one hand, and the total concentrations (initial conditions)
$S_T$ and $E_T$ on the other.
For future reference, we also remark that $P$ evolves under
%
\be
 P' = k_2 C ,
\quad\mbox{subject to}\
 P_0 = 0 .
\label{P-IVP}
\ee
%


\subsection{New parameters}
%
Adapting ideas from \cite{SS89},
we define the new parameters
%
\be
\begin{array}{rclcl}
 K_M &=& \dst\frac{k_{-1} + k_2}{k_1} ,
\vspace{2mm}\\
 V_M &=& k_2 E_T ,
\vspace{2mm}\\
 \sigma &=& \dst\frac{S_T}{K_M} ,
\vspace{2mm}\\
 \kappa &=& \dst\frac{k_{-1}}{k_2} ,
\vspace{2mm}\\
 \eps &=& \dst\frac{E_T}{S_T + K_M} .
\end{array}
\label{params-new}
\ee
%
Here, $K_M$ has the units of concentration
and $V_M$ of concentration per unit of time;
the remaining parameters are non-dimensional.
Transformation~\eqref{params-new} is invertible, with
%
\be
\begin{array}{rclcl}
 k_{-1} &=& \dst\frac{\kappa V_M}{\eps (\sigma+1) K_M} ,
\vspace{2mm}\\
 k_1 &=& \dst\frac{(\kappa+1) V_M}{\eps (\sigma+1) K_M^2} ,
\vspace{2mm}\\
 k_2 &=& \dst\frac{V_M}{\eps (\sigma+1) K_M} ,
\vspace{2mm}\\
 S_T &=& \dst\sigma K_M ,
\vspace{2mm}\\
 E_T &=& \dst\eps (\sigma+1) K_M .
\end{array}
\label{params-old}
\ee
%
Expressed in terms of these new parameters,
the IVP~\eqref{SC-ODE} becomes
%
\be
\begin{array}{rclcl}
 S'
&=&
\dst
 \frac{(\kappa+1)V_M}{K_M}
\left[
 -S
+
 \frac{1}{\eps(\sigma+1)K_M} C S
+
 \frac{\kappa}{\eps(\sigma+1)(\kappa+1)} C
\right] ,
\vspace{2mm}\\
 C'
&=&
\dst
 \frac{(\kappa+1)V_M}{K_M}
\left[
\ \ \,
 S
-
 \frac{1}{\eps(\sigma+1)K_M} C S
-
 \frac{1}{\eps(\sigma+1)} C
\right] .
\end{array}
\label{SC-ODE-new}
\ee
%
The full system dynamics are informed
by \textit{all} five parameters,
and the initial conditions are
%
\be
 S_0 = \sigma K_M
\quad\mbox{and}\quad
 C_0 = 0 .
\label{SC-IC-new}
\ee
%
Note, also, that $P$ evolves under the law
%
\be
 P' = \frac{V_M}{\eps (\sigma+1) K_M} C ,
\quad\mbox{subject to}\
 P_0 = 0 .
\label{P-IVP-new}
\ee
%

\subsection{Slow and fast dynamics}
%
Fast and slow evolution laws are properly derived
by first non-dimensionalizing \eqref{SC-ODE} or \eqref{SC-ODE-new}
and then analyzing asymptotically the resulting, non-dimensional system.
At leading order, we can work with the dimensional system instead.

The slow invariant manifold (SIM) is obtained
as the nullcline of $C$ (always at leading order),
%
\be
 C = \eps (\sigma + 1) K_M \frac{S}{S + K_M} .
\label{C-SIM-new}
\ee
%
The leading order dynamics on the SIM (\textit{slow dynamics})
are found by substituting this expression in the ODE for $S$,
%
\be
 S' = -V_M \frac{S}{S + K_M} ,
\quad\mbox{subject to}\
 S_0 = \sigma K_M .
\label{S-IVP-new-red}
\ee
%
Note carefully that these leading order, slow dynamics
are only influenced by three (new) parameters:
$K_M$, $V_M$ and $\sigma$.
The remaining two (new) parameters,
$\eps$ and $\kappa$,
influence the fast dynamics
and higher order terms in the slow dynamics.

This fast dynamics is found, at leading order,
by fixing $S$ to its initial value $\sigma K_M$.
Substituting into the ODE for $C$,
we resolve the transient at leading order:
%
\[
 C'
=
 -\frac{(\kappa+1)V_M}{\eps K_M} (C - \eps\sigma K_M) .
\]
%
Note that these dynamics are informed by the entire quintuple
$(K_M,V_M,\sigma,\kappa,\eps)$.

The above suggests strongly that observations of the slow dynamics
leads to good fits for $(K_M,V_M,\sigma)$
but \emph{not} for $(\eps,\kappa)$.
These two last variables act as
\emph{regular perturbation parameters} for the slow dynamics,
which should demonstrate itself as sloppiness in fitting.
In this sense, the transformation~\eqref{params-new}
straightens out the sloppy directions in parameter space
(i.e. identifies them with $\eps$ and $\kappa$).

\section{Numerical experiments}
%
Typically, an experimentalist will monitor product levels
and deduce $K_M$ and $V_M$ (both in the new parameter set)
by some form of curve fitting \cite{JG11,LB34,MM13}.
These two parameters are characteristic
of the enzymatic reaction at hand
and viewed as effectively describing it.
Indeed, in the slow phase,
the constraint~\eqref{C-SIM-new} is leading order valid,
and renders the evolution law \eqref{P-IVP-new}
%
\be
 P' = V_M \frac{S}{S + K_M} ,
\quad\mbox{subject to}\
 P_0 = 0 .
\label{P-ODE-new-red}
\ee
%
Here, $S$ evolves under \eqref{S-IVP-new-red}.
This dynamic law is the reduced system description and,
indeed, it involves $K_M$ and $V_M$ only.

In a laboratory setting,
the experiment is initialized with $C_0 = P_0 = 0$
and known original concentrations $S_0=S_T$ and $E_0=E_T$.
Note carefully that these initial conditions
belong to the \emph{original parameter set}.
Since measurements are made in the slow phase,
product is effectively produced
according to \eqref{S-IVP-new-red}--\eqref{P-ODE-new-red}.
It is important to understand here that,
once these \emph{original} parameters have been fixed
(i.e., initial conditions assigned),
$K_M$ and $\sigma$ become \emph{constrained} due to \eqref{params-new}.
This has important consequences
for the experimental design below.

\subsection{Experiment \#1}
%
We will first work with the system
\textit{expressed in the new parameter set},
so as to verify that fitting for $K_M$, $V_M$ and $\sigma$
is \textit{not} sloppy.
(Working with the new parameters circumvents the problem
of having $K_M$ and $\sigma$ constrained on each other,
because we specify values of the new---not the original---parameters.)

To that end, the simulation protocol can be the following.

\noindent{\bf 1.}
Fix values $K^*_M$, $V^*_M$ and $\sigma^*$ for the three parameters
(e.g., close to one).

\noindent{\bf 2.}
Fix a value of $\eps$ (e.g. $10^{-2}$) and a value of $\kappa$
(e.g. close to $10$).

\noindent{\bf 3.}
Specify the time instants at which the product will be measured.
An appropriate timescale for slow motion
is $t_s = (\sigma^*+1)K^*_M/V^*_M$ \cite{SS89},
so these instants could be $\{t_i = i t_s/5\}_{i=1:N}$
for $N \approx 20$.

\noindent{\bf 4.}
Simulate the system \eqref{SC-ODE-new}--\eqref{P-IVP-new}
for these parameter values and monitor the product $P$
at $\{t_i\}_{i=1:N}$.
This set of values (possibly with some noise)
forms the data set$\{P^*_i\}_{i=1:N}$ we will fit to.

\noindent{\bf 5.}
Run your minimization algorithm,
with the cost function set to
%
\[
 \mathcal{C}(K_M,V_M,\sigma)
=
 \sum_{i=1}^N
 \vert
 P(t_i ; K_M,V_M,\sigma,\eps^*,\kappa^*)
-
 P^*_i
 \vert^2 .
\]
%
Plainly, $P(t ; K_M,V_M,\sigma,\eps^*,\kappa^*)$
is generated by \eqref{SC-ODE-new}--\eqref{P-IVP-new}.

\noindent{\bf 6.}
Check the results for sloppiness (e.g. use Hessian);
hopefully find none.

\subsubsection{Results from Experiment 1}

Again, we expect no sloppiness as the three parameters $K_M, V_M$ and $\sigma$ each independently influence the slow dynamics. However, using the recommended parameter values of $K_M =  V_M = \sigma = 1$, we find that eigenvalues of the Hessian span many orders of magnitude. Each entry $H_{ij}$ of the Hessian was approximated using centered finite differences. Plots of the three resulting eigenvalues versus the finite difference stepsize are given in Figs. (1) to (3) below. \\

\textbf{Note that both a scaling and offset are often given in the y-axis, e.g. the y-values in Fig. (2) range from 1.977642 + 0.5E-6 to 1.977642 + 5.0E-6} \\

\begin{figure}[h!]
  \centering
  \includegraphics[width=1.0\textwidth]{eig0}
  \caption{First eigenvalue versus stepsize}
  \label{fig1}
\end{figure}

\begin{figure}[h!]
  \centering
  \includegraphics[width=0.8\textwidth]{eig1}
  \caption{Second eigenvalue versus stepsize}
  \label{fig1}
\end{figure}

\begin{figure}[h!]
  \centering
  \includegraphics[width=0.8\textwidth]{eig2}
  \caption{Third eigenvalue versus stepsize}
  \label{fig1}
\end{figure}

In the areas of convergence, the eigenvalues range from $\lambda_1 \approx 2.45E-3$ to $\lambda_3 \approx 47$.

\clearpage
\newpage

\subsection{Experiment \#2}
%
The second experiment is similar to \#1,
but the optimization is done over all five (new) parameters:
$K_M$, $V_M$, $\sigma$, $\kappa$ and $\eps$.
To that end, steps $1-4$ remain the same.
At step $5$, the optimization procedure is adapted
to include $(\eps,\kappa)$.
Sloppiness should be evident here, including at the Hessian level:
the sloppy directions are contained in
the $(\eps,\kappa)-$coordinate plane.

\subsubsection{Results from Experiment 2}
In review, here we investigate the full, five parameter model. Because only $K_M, V_M$ and $\sigma$ affect the slow dynamics, we expect sloppiness in the remaining parameters $\epsilon$ and $\kappa$. Thus, if the points in our objective function are taken on the slow manifold, we expect the contributions of $\epsilon$ and $\kappa$ to be insignificant. Here, $K_M =  V_M = \sigma = 1$, $\epsilon = 1E-2$ and $\kappa = 10$ as suggested. Plots of the five eigenvalues versus stepsize are given below.

\begin{figure}[h!]
  \centering
  \includegraphics[width=1.0\textwidth]{alleig0}
  \caption{First eigenvalue versus stepsize}
  \label{fig1}
\end{figure}

\begin{figure}[h!]
  \centering
  \includegraphics[width=0.8\textwidth]{alleig1}
  \caption{Second eigenvalue versus stepsize}
  \label{fig1}
\end{figure}

\begin{figure}[h!]
  \centering
  \includegraphics[width=0.8\textwidth]{alleig2}
  \caption{Third eigenvalue versus stepsize}
  \label{fig1}
\end{figure}

\begin{figure}[h!]
  \centering
  \includegraphics[width=0.8\textwidth]{alleig3}
  \caption{Fourth eigenvalue versus stepsize}
  \label{fig1}
\end{figure}

\begin{figure}[h!]
  \centering
  \includegraphics[width=0.8\textwidth]{alleig4}
  \caption{Fifth eigenvalue versus stepsize}
  \label{fig1}
\end{figure}

We do find that the two smallest eigenvalues lie below those found in Experiment (1), but only the first is significantly smaller.

\end{document}
