\documentclass[11pt]{article}

\usepackage{graphicx, subcaption, amsfonts, amsmath, amsthm, empheq, setspace, lscape, xcolor}
\usepackage[top=1in, bottom=1in, left=1in, right=1in]{geometry}

% define some commands
% command to box formula
\newcommand*\widefbox[1]{\fbox{\hspace{2em}#1\hspace{2em}}}
\newlength\dlf
\newcommand\alignedbox[2]{
  % Argument #1 = before & if there were no box (lhs)
  % Argument #2 = after & if there were no box (rhs)
  &  % Alignment sign of the line
  {
    \settowidth\dlf{$\displaystyle #1$}  
    % The width of \dlf is the width of the lhs, with a displaystyle font
    \addtolength\dlf{\fboxsep+\fboxrule}  
    % Add to it the distance to the box, and the width of the line of the box
    \hspace{-\dlf}  
    % Move everything dlf units to the left, so that & #1 #2 is aligned under #1 & #2
    \boxed{#1 #2}
    % Put a box around lhs and rhs
  }
}

\newcommand\ER{Erd\H{o}s-R'{e}nyi}
\newcommand{\Forall}{\; \forall \;}
\DeclareMathOperator*{\argmin}{\arg\!\min}

% change captions
\captionsetup{width=0.8\textwidth}
\captionsetup{labelformat=empty,labelsep=none}

% set paragraph indent length
\setlength\parindent{0pt}

% set folder for imported graphics
\graphicspath{ {./figs/} }

\title{Sloppiness in simple QSSA reaction system}
\date{}

\begin{document}
\maketitle

\section{The system}

We investigate the simple reaction system

\begin{align*}
  A \xrightleftharpoons[k_{-1}]{k_1} B, \; B \xrightarrow[]{k_2} C
\end{align*}

under the assumption that species $B$ is in quasi equilibrium. This requires

\begin{align*}
  \frac{dC_B}{dt} &= k_1 C_A - (k_{-1} + k_2) C_B = 0 \\
  &\rightarrow C_B = C_A \frac{k_1}{k_{-1} + k_2}
\end{align*}

leading to the analytical, QSSA solution for $C_C$

\begin{align*}
  C_C = C_{A_0}(1 - e^{-\frac{k_1 k_2}{k_{-1} + k_2} t})
\end{align*}

One of the beautiful features of this system is that its linearity yields an exact, if messier, analytical solution

\begin{align*}
  C_C = C_{A_0}(\frac{k_1 k_2}{\alpha \beta} + \frac{k_1 k_2}{\alpha(\alpha - \beta)} e^{-\alpha t} - \frac{k_1 k_2}{\beta(\alpha - \beta)} e^{-\beta t})
\end{align*}

where

\begin{align*}
  \alpha &= \frac{1}{2}(k_1 + k_{-1} + k_2 + \sqrt{(k_1 + k_{-1} + k_2)^2 - 4 k_1 k_2}) \\
  \beta &=  \frac{1}{2}(k_1 + k_{-1} + k_2 - \sqrt{(k_1 + k_{-1} + k_2)^2 - 4 k_1 k_2})
\end{align*}

In addition, we see the effective parameter

\begin{align*}
  k_{eff} = \frac{k_1 k_2}{k_{-1} + k_2}
\end{align*}

will create nonlinear level sets in parameter space instead of simple planes.

\section{DMAPS on sloppy parameter combinations}

The base system was taken to have the parameters and initial conditions

\begin{align*}
  k_1 &= 1.0 \\
  k_{-1} &= 1000.0 \\
  k_2 &= 1000.0 \\
  C_{A_0} = 1.0
\end{align*}

As $k_{-1} = k_2 \gg k_1$ this puts the system in the QSSA regime. Then points were sampled in the three-dimensional parameter space, and kept if their objective function evaluated beneath a certain tolerance (here, a point was kept if $\| C - \hat{C}\|^2 < \epsilon$ with $\epsilon = 0.002$ and where $C$ is a vector containing the concentration of $C$ at evenly spaced times using the base parameters, and $\hat{C}$ is a vector of concentrations evaluated at the test parameters. This resulted in the following nonlinear dataset

\begin{figure}[htbp]
  \centering
  \includegraphics[width=\linewidth]{dataset}
  \caption{Dataset of sloppy parameter combinations}
\end{figure}

As expected, it maps out a two-dimensional surface in parameter space over which $k_{eff}$ is nearly constant ($k_{eff} \in (0.4, 1.0)$ in the dataset above).

When DMAPS was applied, the first two eigenvectors parameterized the surface as hoped. This is shown in the two figures below.

\begin{figure}[htbp]
  \centering
  \includegraphics[width=\linewidth]{dmap1}
  \caption{Coloring the dataset by the first DMAP parameter/coordinate value}
\end{figure}

\begin{figure}[htbp]
  \centering
  \includegraphics[width=\linewidth]{dmap2}
  \caption{Coloring the dataset by the second DMAP parameter/coordinate value}
\end{figure}


% \bibliographystyle{plain}
% \bibliography{literature.bib}

\end{document}
