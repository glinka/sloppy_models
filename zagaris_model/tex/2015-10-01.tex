\documentclass[11pt]{article}

\usepackage{graphicx, subcaption, amsfonts, amsmath, amsthm, empheq, setspace, lscape, xcolor}
\usepackage[top=1in, bottom=1in, left=1in, right=1in]{geometry}

% define some commands
% command to box formula
\newcommand*\widefbox[1]{\fbox{\hspace{2em}#1\hspace{2em}}}
\newlength\dlf
\newcommand\alignedbox[2]{
  % Argument #1 = before & if there were no box (lhs)
  % Argument #2 = after & if there were no box (rhs)
  &  % Alignment sign of the line
  {
    \settowidth\dlf{$\displaystyle #1$}  
    % The width of \dlf is the width of the lhs, with a displaystyle font
    \addtolength\dlf{\fboxsep+\fboxrule}  
    % Add to it the distance to the box, and the width of the line of the box
    \hspace{-\dlf}  
    % Move everything dlf units to the left, so that & #1 #2 is aligned under #1 & #2
    \boxed{#1 #2}
    % Put a box around lhs and rhs
  }
}

\newcommand\ER{Erd\H{o}s-R'{e}nyi}
\newcommand{\Forall}{\; \forall \;}
\DeclareMathOperator*{\argmin}{\arg\!\min}

% change captions
\captionsetup{width=0.8\textwidth}
\captionsetup{labelformat=empty,labelsep=none}

% set paragraph indent length
\setlength\parindent{0pt}

% set folder for imported graphics
\graphicspath{ {./figs/} }

\title{Nonlinearly transformed parameters}
\author{}

\begin{document}
\maketitle


To create nonlinear sets of sloppy parameter combinations, we transform the parameters from the original model $\lambda$ and $\epsilon$ into $c_1 = f_1(\lambda, \epsilon)$ and $c_2=f_2(\lambda, \epsilon)$ through invertible $f_1$ and $f_2$. To transform the rectangular region $\lambda \in [0, L]$ and $\epsilon \in [0, E]$, the following was proposed:

\begin{align*}
  c_1 &= \sqrt{\frac{\epsilon}{E} + \frac{1}{S}\frac{\lambda}{L}} \cos(2\pi S \frac{\epsilon}{E}) \\
  c_2 &= \sqrt{\frac{\epsilon}{E} + \frac{1}{S}\frac{\lambda}{L}} \sin(2\pi S \frac{\epsilon}{E})
\end{align*}

where $S$ is a constant affecting the size of the swirl . This has the inversion

\begin{align*}
  \lambda &= SL(c_1^2 + c_2^2 - \frac{1}{2 \pi S}\tan^{-1}(\frac{c_2}{c_1}) \\
  \epsilon &=  \frac{E}{2 \pi S}\tan^{-1}(\frac{c_2}{c_1})
\end{align*}

We start with a system whose true parameters are

\begin{itemize}
\item a = 0.1
\item b = 0.1
\item $\epsilon$ = 1e-3
\item $\lambda$ = 10.0
\end{itemize}

and essentially sample $\epsilon \in (0, 10]$ and $\lambda \in (0, 10]$ (actually we sample $c_1$ and $c_2$ but they're equivalent). This leads to the following contours in the $c_1/c_2$ plane:

\clearpage

\begin{figure}[htbp]
  \centering
  \includegraphics[width=\linewidth]{s-contours-full}
  \caption{The full plane, colored by objective function value. Darkest red corresponds to a least-squares objective function value of 20, darkest blue to 1e-6
\end{figure}

\begin{figure}[htbp]
  \centering
  \includegraphics[width=\linewidth]{s-contours-disjointed}
  \caption{The same data as above, but only keeping points whose objective function value was less than 1e-2.}
\end{figure}

\begin{figure}[htbp]
  \centering
  \includegraphics[width=\linewidth]{s-contours-disjointed2}
  \caption{A different system at different parameter values created in an attempt to make more-nearly continuous contours.}
\end{figure}





% \bibliographystyle{plain}
% \bibliography{literature.bib}

\end{document}
