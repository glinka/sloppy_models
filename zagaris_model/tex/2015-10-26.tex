\documentclass[11pt]{article}

\usepackage{graphicx, subcaption, amsfonts, amsmath, amsthm, empheq, setspace, lscape, xcolor}
\usepackage[top=1in, bottom=1in, left=1in, right=1in]{geometry}

% define some commands
% command to box formula
\newcommand*\widefbox[1]{\fbox{\hspace{2em}#1\hspace{2em}}}
\newlength\dlf
\newcommand\alignedbox[2]{
  % Argument #1 = before & if there were no box (lhs)
  % Argument #2 = after & if there were no box (rhs)
  &  % Alignment sign of the line
  {
    \settowidth\dlf{$\displaystyle #1$}  
    % The width of \dlf is the width of the lhs, with a displaystyle font
    \addtolength\dlf{\fboxsep+\fboxrule}  
    % Add to it the distance to the box, and the width of the line of the box
    \hspace{-\dlf}  
    % Move everything dlf units to the left, so that & #1 #2 is aligned under #1 & #2
    \boxed{#1 #2}
    % Put a box around lhs and rhs
  }
}

\newcommand\ER{Erd\H{o}s-R'{e}nyi}
\newcommand{\Forall}{\; \forall \;}
\DeclareMathOperator*{\argmin}{\arg\!\min}

% change captions
\captionsetup{width=0.8\textwidth}
\captionsetup{labelformat=empty,labelsep=none}

% set paragraph indent length
\setlength\parindent{0pt}

% set folder for imported graphics
\graphicspath{ {./figs/} }

\title{Data space of Antonios' model}


\begin{document}
\maketitle

I used the nonlinear transform $\mu = ax^2$ and $\phi = b y^2$ as originally suggested, and measured $y$ at three separate points on the parabolic slow manifold. This was repeated for a grid of values in the $a,b$ plane. These parameters affect the slow and fast manifold curvature, respecitvely, so we would expect $a$ to be stiff and $b$ sloppy. This would also mean that, in data space, we should find a nearly one-dimensional curve parameterized by $a$. In other words, our ``thick'' ribbon direction would correspond to $a$, while there might be a ``thin'' direction corresponding to changing $b$. This is confirmed in the plots of data space below, the first of which is colored by $a$, the second by $b$. 

I believe the thicker region to the upper right of the second plot is caused by the fact that, at these parameter values, the sampled data aren't quite on the slow manifold yet and thus see some influence of $b$.

\begin{figure}[htbp]
  \centering
  \includegraphics[width=\linewidth]{data-space-a}
  \caption{$\{y(t_1), y(t_2), y(t_3)\}$ sampled at various values of $a$ and $b$. Here colored by $a$-value. The values in data space change with $a$ as expected.}
\end{figure}

\begin{figure}[htbp]
  \centering
  \includegraphics[width=\linewidth]{data-space-b}
  \caption{$\{y(t_1), y(t_2), y(t_3)\}$ sampled at various values of $a$ and $b$. Here colored by $b$-value. The values in data space change are largely invariant to changes in $b$, also as expected.}
\end{figure}



% \bibliographystyle{plain}
% \bibliography{literature.bib}

\end{document}
