\documentclass[11pt]{article}

\usepackage{graphicx, subcaption, amsfonts, amsmath, amsthm, empheq, setspace, lscape}
\usepackage[top=1in, bottom=1in, left=1in, right=1in]{geometry}

% define some commands
% command to box formula
\newcommand*\widefbox[1]{\fbox{\hspace{2em}#1\hspace{2em}}}
\newlength\dlf
\newcommand\alignedbox[2]{
  % Argument #1 = before & if there were no box (lhs)
  % Argument #2 = after & if there were no box (rhs)
  &  % Alignment sign of the line
  {
    \settowidth\dlf{$\displaystyle #1$}  
    % The width of \dlf is the width of the lhs, with a displaystyle font
    \addtolength\dlf{\fboxsep+\fboxrule}  
    % Add to it the distance to the box, and the width of the line of the box
    \hspace{-\dlf}  
    % Move everything dlf units to the left, so that & #1 #2 is aligned under #1 & #2
    \boxed{#1 #2}
    % Put a box around lhs and rhs
  }
}

\newcommand\ER{Erd\H{o}s-R'{e}nyi}
\newcommand{\Forall}{\; \forall \;}
\DeclareMathOperator*{\argmin}{\arg\!\min}

% change captions
\captionsetup{width=0.8\textwidth}
\captionsetup{labelformat=empty,labelsep=none}

% set paragraph indent length
\setlength\parindent{0pt}

% set folder for imported graphics
\graphicspath{ {./figs/} }

\title{Preliminary DMAP results in $K_M, V_M, S_t$ parameters}
\author{}

\begin{document}
\maketitle

\section*{Summary}
By restricting our attention to $K_M, V_M$, and $S_t$, DMAPS uncovered in its $3^{rd}$ and $6^{th}$ eigenvectors a two-dimensional embedding of the $K-V$ plane, while the less sloppy $S_t$ was uncovered later, in the $12^{th}$ eigenvector.

\section*{Details}
Samples were drawn evenly spaced over a log scale, specifically:

\begin{itemize}
\item $K_M \in [10^{-4}, 10^{-1}]$ 
\item $V_M \in [10^{-4}, 10^{-1}]$ 
\item $S_t \in [10^{-7}, 10^{1}]$ 
\end{itemize}

with one hundred points drawn per parameter for a total of one million unique combinations. Combinations for which $c(K_M, V_M, S_t) < 5.0$ were kept (where $c$ is the cost/objective function), resulting in a dataset of approximately 7,000 points. \\

It is important to note that log-parameters were stored, and not the true parameters. Using the Euclidean distance with this scaling more accurately reflects the distances we're interested in; otherwise, any parameters below a certain threshold would essentially be at distance zero from one another. I don't believe sloppiness arising from hidden effective parameter combinations (a lack of ``structural identifiability'') would require such scaling, but that is not the sort of sloppiness in our current example. \\

This log-parameter dataset was used as the input to DMAPS with a kernel epsilon of $\epsilon = 0.8$. The results are shown in the three figures below. The Figs. (1) and (2) show that the two-dimensional embedding can be nicely colored by $K_M$ and $V_M$ values. Fig. (3) shows that the $12^{th}$ eigenvector captures $S_t$.

\begin{figure}[htbp]
  \centering
  \includegraphics[width=\linewidth]{param_surf}
  \caption{Sloppy parameter dataset}
  \label{fig:K}
\end{figure}

\clearpage

\section*{2D Embeddings}

\begin{figure}[htbp]
  \centering
  \includegraphics[width=\linewidth]{k_coloring}
  \caption{Coloring by $K_M$}
  \label{fig:K}
\end{figure}

\begin{figure}[htbp]
  \centering
  \includegraphics[width=\linewidth]{v_coloring}
  \caption{Coloring by $V_M$}
  \label{fig:V}
\end{figure}

\begin{figure}[htbp]
  \centering
  \includegraphics[width=\linewidth]{s_coloring}
  \caption{Coloring by $S_t$}
  \label{fig:S}
\end{figure}

\clearpage

\section*{3D Embeddings}

The three-dimensional embeddings are quite interesting, and suggest that the sloppiness in these three parameters is contained not in some hyper-ribbon like manifold but in something like two intersecting planes. After a certain value of $V_M$ is reached, we pass into a region of constant $V_M$ and sloppy $S_t$. This could be an artifact of sampling: perhaps only at the lower limit of our sampling of $V_M$ is $S_t$ sloppy, and thus if we extended the sampled region we might see more of a three-dimensional structure develop. This is my next step.

\begin{figure}[htbp]
  \centering
  \includegraphics[width=\linewidth]{k_coloring_3d}
  \caption{Coloring by $K_M$. $K_M$ varies over both planes, mainly in the $\phi_3$ direction.}
  \label{fig:K}
\end{figure}

\begin{figure}[htbp]
  \centering
  \includegraphics[width=\linewidth]{v_coloring_3d}
  \caption{Coloring by $V_M$. $V_M$ mainly varies over the right-hand plane.}
  \label{fig:V}
\end{figure}

\begin{figure}[htbp]
  \centering
  \includegraphics[width=\linewidth]{s_coloring_3d}
  \caption{Coloring by $S_t$. $S_t$ varies over the left-hand plane.}
  \label{fig:S}
\end{figure}

\clearpage

\section*{Dataset Two}

Here we again investigate $K_M, V_M$ and $S_t$, but the region of parameter space we searched is now:

\begin{itemize}
\item $K_M \in [10^{-3}, 10^{3}]$ 
\item $V_M \in [10^{-3}, 10^{3}]$ 
\item $S_t \in [10^{0}, 10^{1}]$ 
\end{itemize}

This produces results fairly complementary to those above, although a one-dimensional embedding of $K_M, V_M$ is initially uncovered. I believe the ratio $\frac{K_M}{V_M}$ may be the truly important parameter here. As we might expect, the more highly constrained $S_t$ parameter is uncovered later in the embedding.

% \bibliographystyle{plain}
% \bibliography{bib.bib}

\end{document}