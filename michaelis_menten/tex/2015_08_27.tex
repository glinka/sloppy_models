\documentclass[11pt]{article}

\usepackage{graphicx, subcaption, amsfonts, amsmath, amsthm, empheq, setspace, lscape}
\usepackage[top=1in, bottom=1in, left=1in, right=1in]{geometry}

% define some commands
% command to box formula
\newcommand*\widefbox[1]{\fbox{\hspace{2em}#1\hspace{2em}}}
\newlength\dlf
\newcommand\alignedbox[2]{
  % Argument #1 = before & if there were no box (lhs)
  % Argument #2 = after & if there were no box (rhs)
  &  % Alignment sign of the line
  {
    \settowidth\dlf{$\displaystyle #1$}  
    % The width of \dlf is the width of the lhs, with a displaystyle font
    \addtolength\dlf{\fboxsep+\fboxrule}  
    % Add to it the distance to the box, and the width of the line of the box
    \hspace{-\dlf}  
    % Move everything dlf units to the left, so that & #1 #2 is aligned under #1 & #2
    \boxed{#1 #2}
    % Put a box around lhs and rhs
  }
}

\newcommand\ER{Erd\H{o}s-R'{e}nyi}
\newcommand{\Forall}{\; \forall \;}
\DeclareMathOperator*{\argmin}{\arg\!\min}

% change captions
\captionsetup{width=0.8\textwidth}
\captionsetup{labelformat=empty,labelsep=none}

% set paragraph indent length
\setlength\parindent{0pt}

% set folder for imported graphics
\graphicspath{ {./figs/} }

\title{M.M. contours and a new model}

\begin{document}
\maketitle

\section{M.M. contour}

After iterating through a number of methods, it was decided to use a continuation algorithm to uncover contours in the $S_t$, $V$ plane. This was combined with a second-order continuation method along a geodesic of the contour surface to step through $K$ values, yielding the ellipsoid shown below. The surface matches our expectations as $S_t$ is tightly constrained, confined to (1.99, 2.01), while the ratio $\frac{K}{V}$ exhibits sloppiness, allowing ranges in $K$ of (1.6, 2.5) and in $V$ of (0.9, 1.15). The final figure illuminates this by increasing the range of $S_t$ in the plot to match that of $V$.

\begin{figure}[ht]
  \centering
  \includegraphics[width=\textwidth]{figs/K_V_St_1}
\end{figure}

\begin{figure}[ht]
  \centering
  \includegraphics[width=\textwidth]{figs/K_V_St_2}
\end{figure}

% \begin{figure}[ht]
%   \centering
%   \includegraphics[width=\textwidth]{figs/K_V_St_3}
% \end{figure}

% \begin{figure}[ht]
%   \centering
%   \includegraphics[width=\textwidth]{figs/K_V_St_4}
% \end{figure}

\begin{figure}[ht]
  \centering
  \includegraphics[width=\textwidth]{figs/K_V_St_5}
\end{figure}

\clearpage

\section{Zagaris Model}

(In separate pdf)

% \bibliographystyle{plain}
% \bibliography{literature.bib}

\end{document}
