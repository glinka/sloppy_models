\documentclass[11pt]{article}

\usepackage{graphicx, subcaption, amsfonts, amsmath, amsthm, empheq, setspace, lscape}
\usepackage[top=1in, bottom=1in, left=1in, right=1in]{geometry}

% define some commands
% command to box formula
\newcommand*\widefbox[1]{\fbox{\hspace{2em}#1\hspace{2em}}}
\newlength\dlf
\newcommand\alignedbox[2]{
  % Argument #1 = before & if there were no box (lhs)
  % Argument #2 = after & if there were no box (rhs)
  &  % Alignment sign of the line
  {
    \settowidth\dlf{$\displaystyle #1$}  
    % The width of \dlf is the width of the lhs, with a displaystyle font
    \addtolength\dlf{\fboxsep+\fboxrule}  
    % Add to it the distance to the box, and the width of the line of the box
    \hspace{-\dlf}  
    % Move everything dlf units to the left, so that & #1 #2 is aligned under #1 & #2
    \boxed{#1 #2}
    % Put a box around lhs and rhs
  }
}

\newcommand\ER{Erd\H{o}s-R'{e}nyi}
\newcommand{\Forall}{\; \forall \;}
\DeclareMathOperator*{\argmin}{\arg\!\min}

% change captions
\captionsetup{width=0.8\textwidth}
\captionsetup{labelformat=empty,labelsep=none}

% set paragraph indent length
\setlength\parindent{0pt}

% set folder for imported graphics
\graphicspath{ {./figs/} }

\title{Contours in $K, V, S_t$}

\begin{document}
\maketitle

The following figures illustrate the effects of $K, V$ and $S_t$ on the objective function value. The true system parameters were:

\begin{itemize}
\item $K = 2.0$
\item $V = 1.0$
\item $S_t = 2.0$
\item $\epsilon = 1*10^{-3}$
\item $\kappa = 10.0$
\end{itemize}

The objective function was then evaluated over a grid of points in the $K/V$ plane at varying values of $S_t$. The results are shown in the figures below. These reveal that $S_t$ is more tightly constrained than either $K$ or $V$, as expected. Also, the correlation between $K$ and $V$ at large values is clearly visible. Color denotes $\log(c(K, V, S_t))$, i.e. the log of the objective function value, and are kept consistent from figure to figure, e.g. yellow always corresponds to an objective function value of $\approx 1$ and the darkest red is always $c = 10$.

\begin{figure}[htbp]
  \centering
  \includegraphics[width=0.9\linewidth]{{{contours_S1.19935394621_tol19}}}
\end{figure}

\begin{figure}[htbp]
  \centering
  \includegraphics[width=0.9\linewidth]{{{contours_S1.35387618002_tol19}}}
\end{figure}

\begin{figure}[htbp]
  \centering
  \includegraphics[width=0.9\linewidth]{{{contours_S1.52830673266_tol19}}}
\end{figure}

\begin{figure}[htbp]
  \centering
  \includegraphics[width=0.9\linewidth]{{{contours_S1.72521054994_tol19}}}
\end{figure}

\begin{figure}[htbp]
  \centering
  \includegraphics[width=0.9\linewidth]{{{contours_S1.94748303991_tol19}}}
\end{figure}

\begin{figure}[htbp]
  \centering
  \includegraphics[width=0.9\linewidth]{{{contours_S2.19839264886_tol19}}}
\end{figure}

\begin{figure}[htbp]
  \centering
  \includegraphics[width=0.9\linewidth]{{{contours_S2.48162892284_tol19}}}
\end{figure}

\begin{figure}[htbp]
  \centering
  \includegraphics[width=0.9\linewidth]{{{contours_S2.8013567612_tol19}}}
\end{figure}

\begin{figure}[htbp]
  \centering
  \includegraphics[width=0.9\linewidth]{{{contours_S3.16227766017_tol19}}}
\end{figure}

% \bibliographystyle{plain}
% \bibliography{literature.bib}

\end{document}
