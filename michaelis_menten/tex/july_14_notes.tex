\documentclass[11pt]{article}

\usepackage{graphicx, subcaption, amsfonts, amsmath, amsthm, empheq, setspace, lscape}
\usepackage[top=1in, bottom=1in, left=1in, right=1in]{geometry}

% define some commands
% command to box formula
\newcommand*\widefbox[1]{\fbox{\hspace{2em}#1\hspace{2em}}}
\newlength\dlf
\newcommand\alignedbox[2]{
  % Argument #1 = before & if there were no box (lhs)
  % Argument #2 = after & if there were no box (rhs)
  &  % Alignment sign of the line
  {
    \settowidth\dlf{$\displaystyle #1$}  
    % The width of \dlf is the width of the lhs, with a displaystyle font
    \addtolength\dlf{\fboxsep+\fboxrule}  
    % Add to it the distance to the box, and the width of the line of the box
    \hspace{-\dlf}  
    % Move everything dlf units to the left, so that & #1 #2 is aligned under #1 & #2
    \boxed{#1 #2}
    % Put a box around lhs and rhs
  }
}

\newcommand\ER{Erd\H{o}s-R'{e}nyi}
\newcommand{\Forall}{\; \forall \;}
\DeclareMathOperator*{\argmin}{\arg\!\min}

% change captions
\captionsetup{width=0.8\textwidth}
% \captionsetup{labelformat=empty,labelsep=none}

% set paragraph indent length
\setlength\parindent{0pt}

% set folder for imported graphics
\graphicspath{ {./figs/} }

\pagestyle{plain}

\title{Discussion notes}

\begin{document}
\maketitle

The main consideration during our talk was what form the sloppiness in $K$, $V$, and $S$ space took. We determined that there are two regimes corresponding to large and small values of $\frac{K}{S}$. It may help to rewrite Eq. (14) from the May $12^{th}$ notes as

\begin{align*}
  P' = \frac{1}{\frac{1}{V} + \frac{K}{V*S}}
\end{align*}

When $S_t$ is large and we sample early in the trajectory where $S \approx S_t$, and when $K \ll 1$, i.e. $\frac{K}{S} \ll 1$, we see that

\begin{align*}
  P' \approx V
\end{align*}

and thus the contours appear to straighten out at small $K$ to approximately align with the $K$-axis (see Fig. (1)). In the other case when $\frac{K}{S} \gg 1$ we must examine the value of $V$. For $V \gg 1$ we find

\begin{align*}
  P' \approx \frac{V*S}{K}
\end{align*}

and the ratio $\frac{V}{K}$ becomes significant. Again, Fig. (1) exhibits this sloppiness as large values of $V$ lead contours following a linear relationship between $K$ and $V$. On the other hand, if $V \ll 1$

\begin{align*}
  P' \approx 0
\end{align*}

We can summarize this as follows:

\begin{enumerate}
\item $\frac{K}{S} \ll 1 \rightarrow V$ significant
\item $\frac{K}{S} \gg 1$
  \begin{enumerate}
  \item $V \gg 1 \rightarrow \frac{V}{K}$ significant
  \item $V \ll 1 \rightarrow$ no significant parameters
  \end{enumerate}
\end{enumerate}

In any case, we expect the other two parameters $\epsilon$ and $\kappa$, whose main influence is on the fast dynamics, to exhibit sloppiness when $\epsilon \ll 1$. Fig. (2) confirms that this is the case.

\clearpage

Plots of the log-error in prediction of the $P$ timecourse, normalized by the true values of $P$: $\log(\frac{\sum_i (\hat{P(t_i)} - P(t_i))^2}{\sum_i P(t_i)^2})$ where $\hat{P(t_i)}$ is the concentration of $P$ at time $t_i$ based on new parameter values, while $P(t_i)$ is the true value of $P$. First the $K/V$ plane is examined, then the $\epsilon, \kappa$ plane. True parameter values were:

\begin{itemize}

\item $K = 2.0$
\item $V = 1.0$
\item $S_t = 2.0$
\item $\epsilon = 1*10^{-3}$
\item $\kappa = 10.0$
\end{itemize}

\begin{figure}[htbp]
  \centering
  \includegraphics[width=0.9\linewidth]{{{contours_k_v}}}
\end{figure}

\begin{figure}[htbp]
  \centering
  \includegraphics[width=0.9\linewidth]{{{contours_eps_kappa}}}
\end{figure}



% \bibliographystyle{plain}
% \bibliography{literature.bib}

\end{document}
