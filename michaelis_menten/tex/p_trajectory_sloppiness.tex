\documentclass[11pt]{article}

\usepackage{graphicx, subcaption, amsfonts, amsmath, amsthm, empheq, setspace, lscape}
\usepackage[top=1in, bottom=1in, left=1in, right=1in]{geometry}

% define some commands
% command to box formula
\newcommand*\widefbox[1]{\fbox{\hspace{2em}#1\hspace{2em}}}
\newlength\dlf
\newcommand\alignedbox[2]{
  % Argument #1 = before & if there were no box (lhs)
  % Argument #2 = after & if there were no box (rhs)
  &  % Alignment sign of the line
  {
    \settowidth\dlf{$\displaystyle #1$}  
    % The width of \dlf is the width of the lhs, with a displaystyle font
    \addtolength\dlf{\fboxsep+\fboxrule}  
    % Add to it the distance to the box, and the width of the line of the box
    \hspace{-\dlf}  
    % Move everything dlf units to the left, so that & #1 #2 is aligned under #1 & #2
    \boxed{#1 #2}
    % Put a box around lhs and rhs
  }
}

\newcommand\ER{Erd\H{o}s-R'{e}nyi}
\newcommand{\Forall}{\; \forall \;}
\DeclareMathOperator*{\argmin}{\arg\!\min}

% change captions
\captionsetup{width=0.8\textwidth}

% set paragraph indent length
\setlength\parindent{0pt}

% set folder for imported graphics
\graphicspath{ {./figs/} }

\title{ \vspace{-3cm} Visualizing sloppiness through concentration trajectories}

\begin{document}
\maketitle

% Fig. (1) exhibits sloppiness in both $\epsilon$ and $\kappa$, respectively singular- and regular-perturbation parameters of the model. Fig. (2) shows sloppiness in two regular perturbation parameters $\frac{K}{V}$ and $\kappa$. Only one trajectory in Fig. (2) matches somewhat poorly, when $\kappa \gg 1$ and $K \approx V \approx 1$.

% \begin{figure}[ht]
%   \centering
%   \includegraphics[width=\linewidth]{{{p_timecourse_eps_kappa}}}
% \end{figure}

% \begin{figure}[ht]
%   \centering
%   \includegraphics[width=\linewidth]{{{p_timecourse_kv_kappa}}}
% \end{figure}

% \clearpage

True parameter values were:

\begin{itemize}
\item $K = 2.0$
\item $V = 1.0$
\item $S_t = 2.0$
\item $\epsilon = 1*10^{-3}$
\item $\kappa = 10.0$
\end{itemize}


Fig. (1) shows visual discrepancies as $\epsilon$ increases while Fig. (2) shows discrepancies as $\frac{K}{V}$ increases. Note that in Fig. (2), $\frac{K}{V}$ is held constant at $\frac{K}{V} = \frac{10}{3}$; thus, despite the legend suggesting that only $K$ changes, both $K$ and $V$ are being adjusted to maintain this ratio. The legend indicates the corresponding parameter values and errors, where a sort of average $l_2$ norm was used:

\begin{align}
  \textrm{error}(\hat{P}) = \frac{\sqrt{\sum_{i=1}^N (\hat{P_i} - P_i)^2}}{N}
\end{align}

where $\hat{P_i} - P_i$ is the difference between the predicted and true concentration values at $t_i$. \\

\noindent\fbox{%
  \parbox{\textwidth}{%
    Note that while the reported errors may seem small, a rough estimation shows that the values are commensurate with our definition. Assuming $\hat{P_i} - P_i \approx 0.5$, we find that $\textrm{error} \approx \frac{\sqrt{(20*0.5)^2}}{20} = 0.16$ which is not far off the error of $0.11$ reported for the yellow trajectory in Fig. (2)
  }
}


\begin{figure}[ht]
  \centering
  \includegraphics[width=1.1\linewidth]{{{p_timecourse_eps_visual}}}
  \caption{}
\end{figure}

\begin{figure}[ht]
  \centering
  \includegraphics[width=1.1\linewidth]{{{p_timecourse_kv_visual}}}
  \caption{}
\end{figure}

\clearpage

It is possible to generate trajectories above and below the true profile, though it becomes difficult to visualize when adjusting $\epsilon$ because of the small errors incurred when decreasing below the true value. Fig. (3) zooms in on a particular point revealing that the red trajectory lies slightly above the true values, and the yellow below. Fig. (4) adjusts $K$ and $V$ to create different profiles.

\begin{figure}[ht]
  \centering
  \includegraphics[width=1.1\linewidth]{{{p_timecourse_eps_visual_above_below}}}
  \caption{}
\end{figure}

\begin{figure}[ht]
  \centering
  \includegraphics[width=1.1\linewidth]{{{p_timecourse_kv_visual_above_below}}}
  \caption{}
\end{figure}

% \bibliographystyle{plain}
% \bibliography{literature.bib}

\end{document}
